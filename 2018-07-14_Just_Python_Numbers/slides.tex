\documentclass[utf8]{beamer}

\usepackage[T1]{fontenc}
\usepackage[brazil]{babel}
\usepackage{inconsolata}
\usepackage{minted}

\definecolor{bgbluish}{HTML}{000033}
\definecolor{rulegreen}{HTML}{335533}
\definecolor{outerbgcolor}{HTML}{F7FFF0}

\setminted{python3, autogobble,
           breaklines, breakanywhere,
           breaksymbolindentleft=0pt, breaksymbolindentright=0pt,
           breaksymbolsepleft=3pt, breaksymbolsepright=3pt,
           breaksymbolright=...,
           bgcolor=bgbluish, style=paraiso-dark,
           fontsize=\fontsize{10pt}{10pt},
           frame=lines, rulecolor=rulegreen, framerule=.7pt}
\setmintedinline{bgcolor={}, style=paraiso-light}

\mode<presentation>
\usetheme{Warsaw}
\usecolortheme{seahorse}
\setbeamercolor{background canvas}{bg=outerbgcolor}
\beamertemplatenavigationsymbolsempty

\setbeamertemplate{footline}{\leavevmode\hbox{%
  \begin{beamercolorbox}[wd=.35\paperwidth, ht=2.25ex, dp=1ex, center]
                        {author in head/foot}
    \usebeamerfont{author in head/foot}
      Danilo J. S. Bellini \hfill \texttt{@danilobellini}
  \end{beamercolorbox}%
  \begin{beamercolorbox}[wd=.65\paperwidth, ht=2.25ex, dp=1ex, center]
                        {title in head/foot}
    \usebeamerfont{title in head/foot}
      \insertshorttitle \hfill
      Just Python @ Creditas -- SP \hfill
      \insertshortdate \hfill
      \insertframenumber\,/\,\inserttotalframenumber
  \end{beamercolorbox}%
}}

\title{Números no Python!}
\subtitle{Just Python}
\author{Danilo J. S. Bellini \\ \texttt{@danilobellini}}
\date{2018-07-14}


\begin{document}


\begin{frame}
  \titlepage
\end{frame}


\begin{frame}{Objetivo}
  Mostrar os recursos do Python a respeito de números e matemática:
  \begin{columns}[t]

    \column{.5\textwidth}
    \begin{itemize}
      \item
      $5$ tipos de números, sendo $3$ \emph{built-ins}:
      \begin{itemize}
        \item \mintinline{python}{int}
        \item \mintinline{python}{float}
        \item \mintinline{python}{complex}
        \item \mintinline{python}{decimal.Decimal}
        \item \mintinline{python}{fractions.Fraction}
      \end{itemize}

      \item Conversão de números de/para strings
      \item Conversão de objetos em números
    \end{itemize}

    \column{.5\textwidth}
    \begin{itemize}
      \item
      Módulos da biblioteca padrão:
      \begin{itemize}
        \item \mintinline{python}{numbers}
        \item \mintinline{python}{math}
        \item \mintinline{python}{cmath}
        \item \mintinline{python}{decimal}
        \item \mintinline{python}{fractions}
        \item \mintinline{python}{random}
        \item \mintinline{python}{statistics}
        \item \mintinline{python}{array}
        \item \mintinline{python}{struct}
      \end{itemize}
    \end{itemize}

  \end{columns}
\end{frame}


\begin{frame}[fragile]{Literais}
  Um ``literal'' é um objeto que a linguagem permite escrever
  diretamente em um dado tipo (o próprio \emph{token} denota o tipo).
  Não é um literal algo que exige conversão/processamento
  ou armazenamento associado a algum identificador.
  Dos números em Python, apenas os $3$ tipos built-in possuem literais.

  \begin{minted}{python}

    # int
    2
    128_237  # O _ é apenas visual
    12_82_37 # Este é o mesmo número acima

    # float (presença do ".")
    2.3
    234_122.999_112
    float("inf") # Isto não é um literal de float,
                 # mas o "inf" é um literal do tipo string

    # complex (sufixo "j" ou "J")
    2.23j
    2 + 3J
    complex(2, 3) # O número acima, sem ser como literal

  \end{minted}

\end{frame}


\begin{frame}[fragile]{\texttt{int}:
                       Número inteiro de precisão arbitrária}
  Muito mais que $32$, $64$ ou $128$ bits!

  \begin{minted}{python}

    >>> def fatorial(n):
    ...     return n * fatorial(n - 1) if n > 1 else 1
    ...
    >>> fatorial(12)
    479001600
    >>> fatorial(32)
    263130836933693530167218012160000000
    >>> fatorial(132)
    111824865119600430744996307607616902997562475571842633838412167568361169672820118454045730260688510087990927196104962685462595837360336094267205134948250389032461924909766607715924086489297715200000000000000000000000000000000

  \end{minted}

\end{frame}


\begin{frame}[fragile]{Representações do \texttt{int} em outras bases}
  A base padrão para os inteiros literais é $10$.
  Há $3$ \emph{built-ins} específicos
  para representação de inteiros em strings
  em potências de $2$ como base:
  \mintinline{python}{bin},
  \mintinline{python}{oct} e
  \mintinline{python}{hex}.
  A string fornecida é da mesma forma que o Python aceita
  um literal em tais bases.

  \begin{columns}

    \column{.3\textwidth}
    \begin{minted}{python}

      >>> # Binário
      >>> 0b10110
      22
      >>> bin(23)
      '0b10111'
      >>> # Hexadecimal
      >>> 0x16
      22
      >>> hex(23)
      '0x17'
      >>> # Octal
      >>> 0o26
      22
      >>> oct(23)
      '0o27'

    \end{minted}

    \column{.7\textwidth}
    \begin{minted}{python}

      >>> int("2112", base=3) # Base arbitrária
      68
      >>> 2*27 + 1*9 + 1*3 + 2*1 # Check!
      68
      >>> 00
      0
      >>> 01 # Prefixo 0: octal no Python 2
      Traceback (most recent call last):
        File "<stdin>", line 1
          01
           ^
      SyntaxError: invalid token

    \end{minted}

  \end{columns}
\end{frame}


\begin{frame}[fragile]{Operadores matemáticos}
  Os operadores,
  denotados por símbolos como
  \mintinline{python}{+} e \mintinline{python}{-},
  são \emph{unários} quando possuem $1$ argumento/número,
  e \emph{binários} quando possuem $2$.
  Cada operador executa um \emph{dunder}
  (método com \emph{double underscore}) do tipo do número,
  cujo nome está no comentário:

  \begin{minted}{python}

    >>> +2 # __pos__                       # O "@", __matmul__,
    2
    >>> -2 # __neg__                       # não é usado pelas
    -2
    >>> 2 + 2 # __add__ (soma)             # classes do Python
    4
    >>> 2 * 4 # __mul__ (multiplicação), precede soma
    8
    >>> 2 ** 4 # __pow__ (potenciação), precede multiplicação
    16
    >>> 17 % 3 # __mod__ (resto da divisão)
    2
    >>> 17 // 3 # __floordiv__
    5
    >>> 17 / 3 # __truediv__
    5.666666666666667

  \end{minted}

\end{frame}


\begin{frame}[fragile]{Divisão}
  O Python 2 possuía um \mintinline{python}{__div__},
  o qual era associado ao operador ``\mintinline{python}{/}''
  na ausência do \mintinline{python}{__future__}.
  Em todo legado que ainda utilizar o Python 2,
  é recomendado utilizar esse \mintinline{python}{__future__},
  a qual deve ser a primeira linha de todo módulo.

  \begin{minted}{python}

    >>> 3 / 2 # __div__ no Python 2
    1
    >>> from __future__ import division
    >>> 3 / 2 # __truediv__
    1.5
    >>> divmod(7, 5) # 7 // 5, 7 % 5
    (1, 2)

  \end{minted}

  O \mintinline{python}{divmod} é um \emph{built-in}
  existente em ambas as versões do Python.

\end{frame}


\begin{frame}[fragile]{\emph{Built-in} \texttt{range}}
  Esse é um exemplo famoso de uso de inteiros,
  o \mintinline{python}{range} devolve um objeto que gera
  tardiamente valores inteiros dentro de uma faixa,
  e comporta-se como uma sequência.
  No Python 2, esse \emph{built-in} devolvia uma lista,
  e o \mintinline{python}{xrange} é o que possuía um comportamento
  mais próximo do \mintinline{python}{range} do Python 3.

  \begin{minted}{python}

    >>> range(5) # 1 parâmetro: stop (nunca incluído)
    range(0, 5)
    >>> range(3, 7) # 2 parâmetros: start, stop
    range(3, 7)
    >>> list(range(3, 7))
    [3, 4, 5, 6]
    >>> range(3, 7)[2:] # Pode aplicar slices
    range(5, 7)
    >>> list(range(3, 7, 2)) # Terceiro parâmetro: step
    [3, 5]
    >>> list(range(-5, -12, -2)) # Negativos!
    [-5, -7, -9, -11]

  \end{minted}

  Djikstra prefere esse modelo pois
  \mintinline{python}{len(...) == stop - start}.
\end{frame}


\begin{frame}[fragile]{Módulo \texttt{operator}: operadores}
  Os operadores são funções no módulo \mintinline{python}{operator},
  as quais podem ser utilizadas
  ao invés dos \emph{tokens} (símbolos) dos operadores.
  Isto pode ser útil caso o operador seja um parâmetro.

  \begin{minted}{python}

    >>> import operator
    >>> operator.neg(25) # Operador unário
    -25
    >>> operator.add(3, 5) # Operador binário
    8
    >>> from functools import reduce
    >>> reduce(operator.mul, [5, 3, 2]) # Produtório
    30

  \end{minted}

  Os operadores relativos aos números existem com o mesmo nome do
  respectivo \emph{dunder}.
\end{frame}


\begin{frame}[fragile]{Operadores booleanos}
  A igualdade de números independe do tipo!

  \begin{minted}{python}

    >>> 7 == 2 # __eq__
    False
    >>> 18 != 18. # __ne__ de int e float
    False
    >>> 7 > 3 # __gt__
    True
    >>> 7 >= 7. # __ge__ de int e float
    True
    >>> 8 < 7 # __lt__
    False
    >>> 5 <= 18 # __le__
    True
    >>> 1 + 0j
    (1+0j)
    >>> 1 + 0j == 1
    True

  \end{minted}

\end{frame}


\begin{frame}[fragile]{Operadores lógicos ou \emph{bitwise}}
  Esses operadores são específicos para inteiros,
  aplicados nos bits pensando no número em binário.

  \begin{minted}{python}

    >>> ~38 # __invert__, ~valor == -(valor + 1)
    -39
    >>> 13 & 7 # __and__, 0b1101 | 0b0111 == 0b0101
    5
    >>> 13 | 7 # __or__, 0b1101 | 0b0111 == 0b1111
    15
    >>> 13 ^ 7 # __xor__, 0b1101 ^ 0b0111 == 0b1010
    10
    >>> 7 << 2 # __lshift__, move 2 bits p/ a esquerda
    28
    >>> 7 >> 1 # __rshift__, move 2 bits p/ a direita
    3

  \end{minted}

\end{frame}


\begin{frame}[fragile]{\texttt{float}:
                       Ponto flutuante radix $2$
                       IEEE 754 de precisão dupla}
  São números escritos da forma \texttt{1.xxxx * 2 ** y},
  além do sinal.
  A representação em base decimal costuma ser aproximada,
  e não existe literal de \mintinline{python}{float} em outra base.

  \begin{minted}{python}

    >>> 0. # Basta 1 dígito decimal e o ponto
    0.0
    >>> -.0 # Há um zero positivo e um negativo!
    -0.0
    >>> 0. == -0. # Só curiosidade, nunca use == com float!
    True
    >>> 1e-2 # Notação científica
    0.01
    >>> 1E+5 # O "+" é opcional, "e"/"E" são iguais
    100000.0
    >>> float.hex(3.) # Método hex, 3 == 0b1.1 * 2
    '0x1.8000000000000p+1'

  \end{minted}

\end{frame}


\begin{frame}[fragile]{Cuidados com o \texttt{float}}
  O número de bits da mantissa (quantidade de \texttt{x}) é fixo,
  mas números como $1/3$ e $1/5$ são dízimas periódicas em binário.

  \begin{minted}{python}

    >>> (1/3).hex()
    '0x1.5555555555555p-2'
    >>> (1/5).hex()
    '0x1.999999999999ap-3'
    >>> 1/3 - 1/2 + 1/6
    -2.7755575615628914e-17
    >>> 1/3 - 1/2 == 1/6
    False

  \end{minted}

\end{frame}


\begin{frame}[fragile]{\texttt{math.isclose}}
  Use \mintinline{python}{math.isclose} ao invés de igualdade
  para comparar números quando estiverem em ponto flutuante.
  A comparação é:
  \[ |a-b| \le max(tol_{rel} \cdot max(|a|, |b|), tol_{abs}) \]

  Em Python, c/ os nomes dos argumentos nominados de tolerância:
  \begin{center}
    \mintinline{python}{
      abs(a-b) <= max(rel_tol * max(abs(a), abs(b)), abs_tol)
    }
  \end{center}

  \begin{minted}{python}

    >>> 1/2 - 1/3 == 1/6
    False
    >>> import math
    >>> math.isclose(1/2 - 1/3, 1/6)
    True
    >>> math.isclose(27, 26.1)
    False
    >>> math.isclose(27, 26.1, abs_tol=1)
    True
    >>> math.isclose(2.7e30, 2.6999e30, abs_tol=1)
    False
    >>> math.isclose(2.7e30, 2.6999e30, rel_tol=.0001)
    True

  \end{minted}
\end{frame}


\begin{frame}[fragile]{Informações no \texttt{sys.float\_info}}

  \begin{minted}{python}

    >>> import sys
    >>> sys.float_info.mant_dig # Dígitos da mantissa
    53
    >>> sys.float_info.max_exp # Maior expoente do 2
    1024
    >>> sys.float_info.min_exp # Menor expoente do 2
    -1021
    >>> sys.float_info.radix
    2
    >>> sys.float_info.max # Maior float representável
    1.7976931348623157e+308
    >>> sys.float_info.min # Menor float representável
    2.2250738585072014e-308
    >>> sys.float_info.max + 1 # O 1 é descartável
    1.7976931348623157e+308
    >>> sys.float_info.max + 1e292 # Passou do máximo!
    inf

  \end{minted}

\end{frame}


\begin{frame}[fragile]{\texttt{float.is\_integer}}
  Uma forma de verificar se o ponto flutuante
  representa um número inteiro.

  \begin{minted}{python}

    >>> (.5 + .5).is_integer()
    True
    >>> (1/3) * 5 * (3/5) # Deveria ser 1
    0.9999999999999999
    >>> ((1/3) * 5 * (3/5)).is_integer() # Gotcha!
    False

  \end{minted}

  O ideal é comparar com \mintinline{python}{math.isclose}.
\end{frame}


\begin{frame}[fragile]{Operadores com \texttt{float}}
  Os mesmos operadores matemáticos e booleanos
  podem ser usados com ponto flutuante.

  \begin{minted}{python}

    >>> 2.27 > 2.25
    True
    >>> -2.27 > -2.25
    False
    >>> 2 ** -4 # __pow__ de inteiros pode devolver float!
    0.0625
    >>> 2 ** .25 # Raiz quarta usando __pow__
    1.189207115002721
    >>> import math
    >>> math.sqrt(2) # Raiz quadrada
    1.4142135623730951
    >>> 2 ** .5 # Alternativa (prefira o math.sqrt)
    1.4142135623730951

  \end{minted}
\end{frame}


\begin{frame}[fragile]{Divisão com \texttt{float} e \texttt{round}}
  \begin{minted}{python}

    >>> 2.7 / 1.2 # Divisão
    2.2500000000000004
    >>> 2.7 // 1.2
    2.0
    >>> round(2.7 / 1.2) # Versões antigas eram como o //
    2
    >>> round(2.7 / 1.2, 0) # Chama float.__round__(x, ndigits)
    2.0
    >>> round(2.7 / 1.2, 1)
    2.3
    >>> 2.7 % 1.2 # Resto da divisão
    0.30000000000000027
    >>> 2.7 % 1 # Parte fracionária
    0.7000000000000002

  \end{minted}
\end{frame}


\begin{frame}[fragile]{\texttt{nan}, \texttt{inf} e
                       verificação no \texttt{math}}
  Ponto flutuante possui $3$ números especiais:
  \mintinline{python}{nan},
  \mintinline{python}{inf} e
  \mintinline{python}{-inf}.

  \begin{minted}{python}

    >>> float("nan") == float("nan") # not a number
    False
    >>> float("inf") == float("inf") + 1
    True
    >>> import math
    >>> math.isnan(float("nan"))
    True
    >>> math.isnan(float("inf"))
    False
    >>> math.isinf(float("nan"))
    False
    >>> math.isinf(float("inf"))
    True
    >>> math.isfinite(float("inf"))
    False
    >>> math.isfinite(float("nan"))
    False
    >>> math.isfinite(1e300)
    True

  \end{minted}
\end{frame}


\begin{frame}[fragile]{\texttt{complex}:
                       Complexos formados por \texttt{float}}
  O complexo é um número na forma \mintinline{python}{a + b * 1j},
  em que o sufixo ``j'' do literal denota a unidade complexa,
  e suas partes \mintinline{python}{a} e \mintinline{python}{b}
  são \mintinline{python}{float}.

  \begin{minted}{python}

    >>> complex(2, 3) # Parecem int, mas ...
    (2+3j)
    >>> complex(2, 3).real
    2.0
    >>> complex(2, 3).imag
    3.0
    >>> (2 + 3j) + (1 - 3j) # Tem os operadores esperados
    (3+0j)
    >>> (2 + 3j) * (1 - 3j) # Incluindo multiplicação!
    (11-3j)
    >>> import cmath
    >>> cmath.e ** (1j * cmath.pi) # exp(pi * 1j) == -1
    (-1+1.2246467991473532e-16j)
    >>> 1j > 2j # Não possuem relação de ordem
    Traceback (most recent call last):
      File "<stdin>", line 1, in <module>
    TypeError: '>' not supported between instances of 'complex' and 'complex'

  \end{minted}

\end{frame}


\begin{frame}[fragile]{\texttt{decimal.Decimal}:
                       Ponto flutuante radix $10$}
  SIM! É ponto flutuante! E a precisão não é ilimitada!
  Mas a base é $10$, os valores são exatos nessa base.
  Útil para trabalhar com valores monetários.

  \begin{minted}{python}

    >>> import decimal
    >>> decimal.getcontext().prec # Precisão (dígitos)
    28
    >>> context = decimal.getcontext()
    >>> context.prec = 5
    >>> decimal.setcontext(context) # Define precisão
    >>> d = decimal.Decimal("18.7654")
    >>> d # Mantém a atribuição
    Decimal('18.7654')
    >>> d * 2 # Utiliza o contexto antes de fazer contas
    Decimal('37.531')
    >>> round(d * 2, 2) # Chama Decimal.__round__
    Decimal('37.53')

  \end{minted}

\end{frame}


\begin{frame}[fragile]{\texttt{fractions.Fraction}:
                       Ponto flutuante radix $10$}
  Frações de números inteiros, já simplificada.

  \begin{minted}{python}

    >>> from fractions import Fraction
    >>> Fraction(5040, 120)
    Fraction(42, 1)
    >>> Fraction(1, 2) - Fraction(1, 3)
    Fraction(1, 6)
    >>> round(Fraction(1, 6), 3) # Força representação decimal
    Fraction(167, 1000)
    >>> 1 / 6 # 0.167, se representado com 3 dígitos após "."
    0.16666666666666666
    >>> Fraction(7, 2).numerator
    7
    >>> Fraction(7, 2).denominator
    2

  \end{minted}

\end{frame}


\begin{frame}[fragile]{\texttt{abs}: valor absoluto}
  Remove o sinal do número,
  devolve a magnitude no caso de um complexo.
  Chama o método \mintinline{python}{__abs__} do tipo.

  \begin{minted}{python}

    >>> abs(-17)
    17
    >>> abs(-2.7)
    2.7
    >>> abs(4 + 3j)
    5.0
    >>> import decimal, fractions
    >>> abs(decimal.Decimal("-8.231"))
    Decimal('8.231')
    >>> abs(fractions.Fraction(8, -3))
    Fraction(8, 3)

  \end{minted}

\end{frame}


\begin{frame}[fragile]{\texttt{bool}}
  Peraê, booleanos não são números!
  Mas para todos os tipos de números, a conversão para booleano é:

  \begin{itemize}
    \item \mintinline{python}{False} quando zero
    \item \mintinline{python}{True} caso contrário
  \end{itemize}

  \begin{columns}

    \column{.4\textwidth}
    \begin{minted}{python}

      >>> bool(-.0) # float
      False
      >>> bool(.0000001)
      True
      >>> bool(float("nan"))
      True
      >>> bool(float("inf"))
      True
      >>> bool(0j) # complex
      False
      >>> bool(0j + .0002)
      True
      >>> bool(.002j)
      True

    \end{minted}

    \column{.6\textwidth}
    \begin{minted}{python}

      >>> bool(0) # int
      False
      >>> bool(42)
      True
      >>> from decimal import Decimal
      >>> bool(Decimal(0))
      False
      >>> bool(Decimal("0.0002"))
      True
      >>> from fractions import Fraction
      >>> bool(Fraction(0, 5))
      False
      >>> bool(Fraction(1, 2**90))
      True

    \end{minted}

  \end{columns}
\end{frame}


\begin{frame}[fragile]{Hash}
  Todo número é imutável e \emph{hashable}.
  Por conta da igualdade com o \mintinline{python}{__eq__},
  o hash de um número não depende do tipo.

  \begin{minted}{python}

    >>> hash(0) # Para inteiros, é o próprio número
    0
    >>> hash(123)
    123
    >>> hash(55.)
    55
    >>> hash(float("inf")) # Valor de sys.hash_info.inf
    314159
    >>> hash(float("nan")) # Valor de sys.hash_info.nan
    0
    >>> hash(.55)
    1268213655067531776
    >>> float("inf") == 314159
    False
    >>> {float("inf"): 1, 314159: 2, # Just for fun
    ...  float("nan"): 3, 0: 4, 0.: 5, 0j: 6}
    {inf: 1, 314159: 2, nan: 3, 0: 6}

  \end{minted}

\end{frame}


\begin{frame}[fragile]{\texttt{math} e \texttt{cmath}}
  Esses módulos possuem funções trigonométricas, logaritmos
  e outras funções matemáticas.
  O \mintinline{python}{cmath} trabalha com complexos, enquanto o
  \mintinline{python}{math} lida principalmente com
  \mintinline{python}{float}.
  Há no \mintinline{python}{math} alguns recursos para inteiros
  (atualmente se discute migrá-los para um novo módulo
  \mintinline{python}{imath}).

  \begin{minted}{python}

    >>> import math, cmath
    >>> cmath.sin(math.pi / 6) # sin(30 degrees)
    (0.49999999999999994+0j)
    >>> math.sin(math.pi / 6)
    0.49999999999999994
    >>> cmath.exp(1j * cmath.pi) # exp(pi * 1j)
    (-1+1.2246467991473532e-16j)
    >>> math.exp(1j * math.pi)
    Traceback (most recent call last):
      File "<stdin>", line 1, in <module>
    TypeError: can't convert complex to float

  \end{minted}

  No próximo slide constam todas as funções do
  módulo \mintinline{python}{cmath}.

\end{frame}


\begin{frame}{Funções matemáticas dos módulos
                       \texttt{math} e \texttt{cmath}}
  \begin{columns}

    \column{.5\textwidth}
    \begin{itemize}
      \item \mintinline{python}{acos}: Arco-cosseno
      \item \mintinline{python}{acosh}: Arco-cosseno hiperbólico
      \item \mintinline{python}{asin}: Arco-seno
      \item \mintinline{python}{asinh}: Arco-seno hiperbólico
      \item \mintinline{python}{atan}: Arco-tangente
      \item \mintinline{python}{atanh}: Arco-tangente hiperbólica
      \item \mintinline{python}{cos}: Cosseno
      \item \mintinline{python}{cosh}: Cosseno hiperbólico
      \item \mintinline{python}{sin}: Seno
      \item \mintinline{python}{sinh}: Seno hiperbólico
      \item \mintinline{python}{tan}: Tangente
      \item \mintinline{python}{tanh}: Tangente hiperbólica
    \end{itemize}

    \column{.5\textwidth}
    \begin{itemize}
      \item \mintinline{python}{exp}: Exponencial
      \item \mintinline{python}{log}: Logaritmo natural
      \item \mintinline{python}{log10}: Logaritmo base $10$
      \item \mintinline{python}{sqrt}: Raiz quadrada
    \end{itemize}

    Específico do \mintinline{python}{cmath}:
    \begin{itemize}
      \item \mintinline{python}{phase}: Fase/argumento do complexo
      \item \mintinline{python}{polar}: Raio e fase do complexo
      \item \mintinline{python}{rect}: Complexo a partir de raio e fase
    \end{itemize}

    Específico do \mintinline{python}{math}:
    \begin{itemize}
      \item \mintinline{python}{degrees}: Radianos p/ graus
      \item \mintinline{python}{radians}: Graus p/ radianos
    \end{itemize}

  \end{columns}
\end{frame}


\begin{frame}[fragile]{Exemplo de execução
                       de funções do \texttt{cmath}}
  \begin{minted}{python}

    >>> from math import sqrt, pi
    >>> from cmath import phase, polar, rect
    >>> abs(1 + 1j) # Módulo
    1.4142135623730951
    >>> phase(1 + 1j) # Fase
    0.7853981633974483
    >>> pi / 4 # A fase é realmente pi / 4
    0.7853981633974483
    >>> polar(1 + 1j) # Cartesiano -> Polar
    (1.4142135623730951, 0.7853981633974483)
    >>> rect(sqrt(2), pi / 4) # Polar -> Cartesiano
    (1.0000000000000002+1j)

  \end{minted}
\end{frame}


\begin{frame}{Funções do \texttt{math} p/ \texttt{float}}
  \begin{columns}
    \column{.58\textwidth}
    Arredondamento:
    \begin{itemize}
      \item \mintinline{python}{trunc}: Trunca p/ mais próximo de 0
      \item \mintinline{python}{floor}: Trunca p/ menor valor
      \item \mintinline{python}{ceil}: Arredonda p/ maior valor
    \end{itemize}

    Operações nativas:
    \begin{itemize}
      \item \mintinline{python}{fabs}: Valor absoluto (float)
      \item \mintinline{python}{fmod}: Resto da divisão da máquina
      \item \mintinline{python}{fsum}: Somatório acurado p/
                                       \mintinline{python}{float}
    \end{itemize}

    Decomposição:
    \begin{itemize}
      \item \mintinline{python}{copysign}: Mistura magnitude/sinal
      \item \mintinline{python}{frexp}: Mantissa/expoente base $2$
      \item \mintinline{python}{modf}: Mantissa e parte inteira
    \end{itemize}

    \column{.42\textwidth}
    Estatística:
    \begin{itemize}
      \item \mintinline{python}{gamma}: Gamma de Euler
      \item \mintinline{python}{erf}: Função erro (integral da normal)
    \end{itemize}

    Outros:
    \begin{itemize}
      \item \mintinline{python}{erfc}: \mintinline{python}{1 - erf(n)}
      \item \mintinline{python}{expm1}: \mintinline{python}{exp(x) - 1}
      \item \mintinline{python}{log1p}: \mintinline{python}{log(1 + x)}
      \item \mintinline{python}{log2}: Logaritmo base $2$
      \item \mintinline{python}{ldexp}: \mintinline{python}{a * 2 ** b}
      \item \mintinline{python}{hypot}:
              \mintinline{python}{sqrt(a*a + b*b)}
      \item \mintinline{python}{pow}: \mintinline{python}{a ** b}
    \end{itemize}

  \end{columns}
\end{frame}


\begin{frame}[fragile]{Exemplo de execução de funções do \texttt{math}}
  \begin{minted}{python}

    >>> import math
    >>> math.gamma(6) # Para inteiros, é o fatorial(n - 1)
    120.0
    >>> math.fmod(-5, 3) # Nem sempre é igual ao %
    -2.0
    >>> (-5) % 3
    1
    >>> math.modf(27.38) # Separa parte fracionária/inteira
    (0.379999999999999, 27.0)
    >>> math.copysign(-27.38, 12) # 1º com sinal do 2º
    27.38
    >>> math.frexp(22) # Decompõe mantissa/expoente base 2
    (0.6875, 5)
    >>> 2 ** 5
    32
    >>> .6875 * 32
    22.0

  \end{minted}
\end{frame}


\begin{frame}[fragile]{Funções p/ inteiros no \texttt{math}}
  Há duas: \mintinline{python}{gcd} e \mintinline{python}{factorial}.

  \begin{minted}{python}

    >>> from math import gcd, factorial
    >>> factorial(7) # 7*6*5*4*3*2*1
    5040
    >>> gcd(18, 12) # Maior divisor comum
    6

  \end{minted}
\end{frame}


\begin{frame}[fragile]{Módulo \texttt{random} para embaralhamento}
  Esse módulo possui rotinas para geração de números pseudo-aleatórios
  fundamentado no Mersenne Twister.
  Segue um exemplo de rotinas para embaralhamento:

  \begin{minted}{python}

    >>> import random
    >>> random.seed(42) # Congela valores p/ re-execuções
    >>> # Seleciona algum dos possíveis valores
    >>> [random.choice("ABCD") for unused in range(10)]
    ['A', 'A', 'C', 'B', 'B', 'B', 'A', 'A', 'D', 'A']
    >>> data = list(range(10))
    >>> random.shuffle(data) # Embaralha in place
    >>> data
    [6, 7, 2, 9, 5, 4, 8, 3, 1, 0]
    >>> random.sample(data, 4) # "Amostra" de data sem repetir
    [1, 8, 9, 3]

  \end{minted}

\end{frame}


\begin{frame}[fragile]{Módulo \texttt{random} para geração de números}
  \begin{minted}{python}

    >>> random.seed(11)
    >>> # Distribuição uniforme:
    >>> random.random() # Gera float 0 <= x < 1
    0.4523795535098186
    >>> a, b = 3, 7
    >>> random.randint(a, b) # Gera int a <= x <= b
    7
    >>> random.randrange(a, b) # Gera int a <= x < b
    6
    >>> random.uniform(a, b) # Gera float entre a e b
    4.807329282398261
    >>> # Há outras distribuições! lognormal, gamma, normal, ...
    >>> random.triangular()
    0.7307841348511

  \end{minted}

  Há ainda uma classe \mintinline{python}{Random} para permitir
  guardar múltiplos estados ao invés de um único estado global.
\end{frame}


\begin{frame}{Medidas de tendência central
              com o módulo \texttt{statistics}}
  \begin{itemize}
    \item \mintinline{python}{mean}: Média
    \item \mintinline{python}{harmonic_mean}: Média harmônica
    \item \mintinline{python}{median}: Mediana (média\footnote{
      Média é dos $2$ valores centrais,
      caso haja um número par de elementos.
    })
    \item \mintinline{python}{median_low}: Mediana (menor)
    \item \mintinline{python}{median_high}: Mediana (maios)
    \item \mintinline{python}{median_grouped}: Mediana do agrupamento
    \item \mintinline{python}{mode}: Moda
  \end{itemize}

  Segue um exemplo ...

\end{frame}


\begin{frame}[fragile]{Medidas de tendência central
                       com o módulo \texttt{statistics}}
  \begin{minted}{python}

    >>> import statistics
    >>> statistics.mean([3, 5, 10]) # int
    6
    >>> statistics.mean([3, 5, 11]) # float!
    6.333333333333333
    >>> statistics.median([3, 5, 6, 11])
    5.5
    >>> statistics.median_low([3, 5, 6, 11])
    5
    >>> statistics.median_high([3, 5, 6, 11])
    6
    >>> statistics.median_grouped([3, 5, 5, 6, 11])
    5.25
    >>> statistics.mean([3, 5, 5, 6, 11])
    6
    >>> statistics.mode([3, 5, 5, 6, 11])
    5

  \end{minted}
\end{frame}


\begin{frame}[fragile]{Medidas de variabilidade
                       com o módulo \texttt{statistics}}
  Com \mintinline{python}{variance} e \mintinline{python}{stdev},
  esse módulo calcula a variância e o desvio padrão amostrais.
  Com o prefixo ``\mintinline{python}{p}'',
  os valores são da população.

  \begin{minted}{python}

    >>> import statistics
    >>> statistics.variance([3, 5, 5, 6, 11])
    9
    >>> statistics.stdev([3, 5, 5, 6, 11])
    3.0
    >>> statistics.pvariance([3, 5, 5, 6, 11])
    7.2
    >>> statistics.pstdev([3, 5, 5, 6, 11])
    2.6832815729997477

  \end{minted}
\end{frame}


\begin{frame}[fragile]{Representação de \texttt{float} em strings}
  Já vimos que podemos converter de strings para números
  usando os construtores das próprias classes.
  Mas e o caminho contrário, dos números chegar nas strings?
  E se quisermos limitar a precisão
  nos números em ponto flutuante? Há $3$ formas:

  \begin{minted}{python}

    >>> "%f" % 15.7
    '15.700000'
    >>> "%g" % 15.7
    '15.7'
    >>> "%.32f" % 15.7 # str.__mod__
    '15.69999999999999928945726423989981'
    >>> "{:.32f}".format(15.7) # str.format
    '15.69999999999999928945726423989981'
    >>> f"{15.7:.32f}" # f-string
    '15.69999999999999928945726423989981'
    >>> number, digits = 15.7, 32
    >>> f"{number:.{digits}f}"
    '15.69999999999999928945726423989981'

  \end{minted}

\end{frame}


\begin{frame}[fragile]{Representação de \texttt{decimal.Decimal}
                       em strings}
  Podemos usar as mesmas formas adotadas
  para converter \mintinline{python}{float},
  mas há um método \mintinline{python}{normalize}
  que permite "cortar zeros à direita".

  \begin{minted}{python}

    >>> from decimal import Decimal
    >>> "%f" % Decimal("15.70")
    '15.700000'
    >>> "%g" % Decimal("15.70")
    '15.7'
    >>> Decimal("15.70")
    Decimal('15.70')
    >>> Decimal("15.70").normalize()
    Decimal('15.7')
    >>> number, digits = Decimal("15.7"), 32
    >>> f"{number:.{digits}f}"
    '15.70000000000000000000000000000000'

  \end{minted}

\end{frame}


\begin{frame}[fragile]{Representação de \texttt{int} em strings}
  \begin{minted}{python}

    >>> "%d" % -232
    '-232'
    >>> "%6d" % -232
    '  -232'
    >>> "%06d" % -232
    '-00232'
    >>> "{:07d}".format(-232)
    '-000232'
    >>> value = -232
    >>> f"{value + 2:08d}"
    '-0000230'

  \end{minted}
\end{frame}


\begin{frame}[fragile]{Módulo \texttt{array}}
  \emph{Container} de números de um único tipo.
  Os métodos lembram os de uma lista.

  \begin{minted}{python}

    >>> from array import array
    >>> ar = array("f", [.1, .2, .3, .4])
    >>> ar.typecode # Ponto flutuante de precisão simples
    'f'
    >>> ar.extend([.5]) # Métodos como do tipo list
    >>> ar.append(1e100) # Maior que o máximo representável
    >>> ar # Exibe o ruído de conversão p/ precisão dupla
    array('f', [0.10000000149011612, 0.20000000298023224, 0.30000001192092896, 0.4000000059604645, 0.5, inf])
    >>> ar.itemsize # Bytes por elemento
    4
    >>> len(ar) # Número de elementos
    6

  \end{minted}

\end{frame}


\begin{frame}{Tipos que podem ser usados com o \texttt{array}}
  \begin{itemize}
    \item \mintinline{python}{"b"}: inteiro de $1$ bytes com sinal
    \item \mintinline{python}{"B"}: inteiro de $1$ bytes sem sinal
    \item \mintinline{python}{"h"}: inteiro de $2$ bytes com sinal
    \item \mintinline{python}{"H"}: inteiro de $2$ bytes sem sinal
    \item \mintinline{python}{"i"}: inteiro de $2$ bytes com sinal
    \item \mintinline{python}{"I"}: inteiro de $2$ bytes sem sinal
    \item \mintinline{python}{"l"}: inteiro de $4$ bytes com sinal
    \item \mintinline{python}{"L"}: inteiro de $4$ bytes sem sinal
    \item \mintinline{python}{"q"}: inteiro de $8$ bytes com sinal
    \item \mintinline{python}{"Q"}: inteiro de $8$ bytes sem sinal
    \item \mintinline{python}{"f"}: ponto flutuante IEEE754 radix $2$
                                    de precisão simples ($4$ bytes)
    \item \mintinline{python}{"d"}: ponto flutuante IEEE754 radix $2$
                                    de precisão dupla ($8$ bytes)
    \item \mintinline{python}{"u"}: unicode de $2$ ou $4$ bytes
  \end{itemize}
\end{frame}


\begin{frame}[fragile]{Módulo \texttt{struct}}
  Esse módulo é usado para comunicação binária,
  convertendo dados de diversos tipos (inclusive números)
  de/para \mintinline{python}{bytes}.

  \begin{minted}[mathescape]{python}

    >>> import struct
    >>> struct.unpack("bb", b"\x05\x06") # Inteiros de $1$ byte
    (5, 6)
    >>> struct.unpack("Bb", b"\xF5\xF6") # Sem/com sinal
    (245, -10)
    >>> struct.pack("hh", 0xC7, 0xF3) # Inteiros de $2$ bytes
    b'\xc7\x00\xf3\x00'
    >>> struct.pack(">hh", 0xC7, 0xF3) # Big endian
    b'\x00\xc7\x00\xf3'
    >>> struct.pack("<hh", 0xC7, 0xF3) # Little endian
    b'\xc7\x00\xf3\x00'
    >>> pair_struct = struct.Struct(">hh") # Lembra re.compile
    >>> pair_struct.pack(0xD8, 0xE2)
    b'\x00\xd8\x00\xe2'
    >>> pair_struct.unpack(b"\x07\x00\x00\xff")
    (1792, 255)

  \end{minted}

  O \mintinline{python}{"hh"} nativo
  pode ser tanto little como big endian.

\end{frame}


\begin{frame}[fragile]{Módulo \texttt{numbers}: ABCs de números}
  Há 5 ABCs de números nesse módulo:
  \mintinline{python}{Integral},
  \mintinline{python}{Rational},
  \mintinline{python}{Real},
  \mintinline{python}{Complex} e
  \mintinline{python}{Number},
  cada um herdando do anterior nessa ordem.

  \begin{minted}{python}

    >>> import numbers
    >>> numbers.Complex.mro()
    [<class 'numbers.Complex'>, <class 'numbers.Number'>, <class 'object'>]
    >>> isinstance(2, numbers.Complex)
    True

  \end{minted}

\end{frame}


\begin{frame}[fragile]{Criando sua ABC c/
                       \texttt{\_\_instancecheck\_\_}}
  \begin{minted}{python}

    >>> class OddMeta(type):
    ...     def __instancecheck__(self, value):
    ...         return isinstance(value, int) and value % 2 == 1
    ...
    >>> class Odd(metaclass=OddMeta):
    ...     pass
    ...
    >>> isinstance(2, Odd)
    False
    >>> isinstance(3, Odd)
    True
    >>> isinstance(4, Odd)
    False
    >>> isinstance(5, Odd)
    True

  \end{minted}
\end{frame}


\begin{frame}[fragile]{Fazendo um método ser aceito como inteiro!}
  Os \emph{dunders}
  \mintinline{python}{__int__},
  \mintinline{python}{__float__} e
  \mintinline{python}{__complex__}
  podem ser usados p/ um objeto
  poder ser convertido em um dos $3$ tipos homônimos.

  \begin{minted}{python}

    >>> class Something(object):
    ...     value = 0
    ...     def __int__(self):
    ...         self.value += 1
    ...         return self.value
    ...
    >>> s = Something()
    >>> 5 + int(s)
    6
    >>> 5 + int(s)
    7
    >>> 5 + int(s)
    8
    >>> 5 + s
    Traceback (most recent call last):
      File "<stdin>", line 1, in <module>
    TypeError: unsupported operand type(s) for +: 'int' and 'Something'

  \end{minted}
\end{frame}


\begin{frame}[fragile]{E com herança?}

  \begin{minted}{python}

    >>> class Sucessor(int):
    ...     def __new__(cls, value):
    ...         return super().__new__(cls, value + 1)
    ...
    >>> Sucessor(7)
    8
    >>> type(Sucessor(7))
    <class '__main__.Sucessor'>

  \end{minted}
\end{frame}


\begin{frame}
  \begin{center}\fontsize{5cm}{2.5cm}\selectfont
    FIM!
  \end{center}
\end{frame}


\end{document}
