\documentclass[utf8]{beamer}

\usepackage[T1]{fontenc}
\usepackage[brazil]{babel}
\usepackage{sourcecodepro}
\usepackage{minted}
\usepackage[absolute, overlay]{textpos}
\usepackage{changepage} % adjustwidth environment
\usepackage{qrcode}
\usepackage{enumitem}
\usepackage{tikz}


\usetikzlibrary{calc, positioning, arrows, arrows.meta}

\definecolor{codebgcolor}{HTML}{FFFFFF}
\definecolor{coderulecolor}{HTML}{5F7FFF}
\definecolor{outerbgcolor}{HTML}{FFFFEE}

\setminted{python3, autogobble,
           breaklines, breakanywhere,
           breaksymbolindentleft=0pt, breaksymbolindentright=0pt,
           breaksymbolsepleft=3pt, breaksymbolsepright=3pt,
           breaksymbolright=...,
           bgcolor=codebgcolor, style=borland,
           fontsize=\fontsize{8pt}{8pt},
           frame=lines, rulecolor=coderulecolor, framerule=.7pt}
\setmintedinline{bgcolor={}}

\mode<presentation>
\usetheme{Warsaw}
\usecolortheme{seahorse}
\setbeamercolor{background canvas}{bg=outerbgcolor}
\beamertemplatenavigationsymbolsempty

\setbeamertemplate{footline}{\leavevmode\hbox{%
  \begin{beamercolorbox}[wd=.34\paperwidth, ht=2.25ex, dp=1ex, center]
                        {author in head/foot}%
    \usebeamerfont{author in head/foot}%
      Danilo J. S. Bellini\hfill\texttt{@danilobellini}%
  \end{beamercolorbox}%
  \begin{beamercolorbox}[wd=.24\paperwidth, ht=2.25ex, dp=1ex, center]
                        {title in head/foot}%
    \usebeamerfont{title in head/foot}%
      \insertshorttitle%
  \end{beamercolorbox}%
  \begin{beamercolorbox}[wd=.35\paperwidth, ht=2.25ex, dp=1ex, center]
                        {author in head/foot}%
    \usebeamerfont{author in head/foot}%
      SECOMP 2024 @ UFSCar
      \raisebox{-0.6ex}{\includesvg{1em}{logo_2024}}
      \insertshortdate%
  \end{beamercolorbox}%
  \begin{beamercolorbox}[wd=.07\paperwidth, ht=2.25ex, dp=1ex, center]
                        {title in head/foot}%
    \usebeamerfont{title in head/foot}%
      \insertframenumber\,/\,\inserttotalframenumber%
  \end{beamercolorbox}%
}}

\setbeamertemplate{background}{%
  \tikz[overlay,remember picture]%
    \node[opacity=0.05, above=-.4cm, anchor=center]
      at (current page.center){%
      \includesvg{.6\paperwidth}{logo_2024_recolored}%
    };%
}

\title{Controle de versão com git}
\subtitle{Minicurso}
\author{Danilo J. S. Bellini \\ \texttt{@danilobellini}}
\date{2024-10-29}

\setlength{\TPHorizModule}{\paperwidth}
\setlength{\TPVertModule}{\paperheight}

\setitemize{label=\usebeamerfont*{itemize item}%
  \usebeamercolor[fg]{itemize item}%
  \usebeamertemplate{itemize item}%
}

\newcommand{\includesvg}[2]{%
  \ifnum\pdfstrcmp{\pdffilemoddate{#2.svg}}%
                  {\pdffilemoddate{#2.pdf}}>0%
    {\immediate\write18%
     {inkscape -z -D --file=#2.svg --export-pdf=#2.pdf --export-latex}%
    }%
  \fi%
  \centering%
  \resizebox{#1}{!}{%
    \input{#2.pdf_tex}%
  }%
}

\newcommand{\repo}[0]{https://github.com/danilobellini/slides-latex}


\begin{document}


\begin{frame}
  \titlepage
  \center
  {\huge SECOMP 2024 @ UFSCar/SP}

  \begin{textblock}{0}(.12, .42)%
    \includesvg{.25\paperwidth}{git_logo}%
  \end{textblock}
  \begin{textblock}{0}(.67, .55)%
    \includesvg{.18\paperwidth}{Git-logo-2007}
  \end{textblock}
\end{frame}


\begin{frame}{Breve história dos sistemas de controle de versão
              e surgimento do \texttt{git}}
  Sistemas de controle de versão:
  \begin{itemize}
    \item CVS (\textit{Concurrent Versions System}), 1986-2008 GPLv1+
    \item BitKeeper 2000-2018, Apache desde 2016-05-09
    \item Subversion (\texttt{svn}), desde 2000-10-20, Apache
    \item Mercurial (\texttt{hg}), desde 2005-04-19, GPLv2+
    \item Git, desde 2005-04-07, GPLv2-only
  \end{itemize}
  Linux (kernel):
  \begin{itemize}
    \item
      Um diretório por release até 2002, não havia controle de versão
    \item
      Usou o BitKeeper (nonfree) desde 2002,
      gratuito como ``incentivo ao software livre''
    \item
      Em 2005, Andrew Tridgell usou engenharia reversa (rede)
      para criar uma alternativa open source ao BitKeeper,
      então Larry McVoy retirou o ``incentivo''
    \item
      Linus Torvalds interrompe o desenvolvimento do Linux
      e inicia o planejamento e o desenvolvimento do git
    \item
      ``Thank You, Larry McVoy'' por Richard Stallman
  \end{itemize}
\end{frame}

\begin{frame}{Por que o nome ``\texttt{git}''?}
  \begin{adjustwidth}{2em}{1em}\emph{\Large
    ``Sou egoísta, e dou nomes relativos a mim
      em todos os meus projetos.'' (Linus Torvalds)
  }\end{adjustwidth}%
  \vfill%
  A palavra \texttt{git} é uma gíria britânica pejorativa.
  O \texttt{README} do projeto diz que o significado
  depende de seu humor:
  \begin{itemize}
    \item
      Arranjo aleatório pronunciável de 3 letras
      que não é utilizado por nenhum comando UNIX comum.
      O fato de poder ser um ``get'' pronunciado erroneamente
      pode ou não ser relevante.
    \item
      Estúpido. Contemptível e desprezível. Simples.
      Escolha com base em seu dicionário de gírias.
    \item
      ``\emph{Global information tracker}'':
      Você está de bom humor, e ele funciona para você.
      Anjos cantam e a luz subitamente preenche a sala.
    \item
      ``\emph{Goddamn idiotic truckload of sh*t}'': quando ele quebra.
  \end{itemize}
  \vfill%
  \textbf{Curiosidade}:
  Como o git é descrito em sua \emph{man page} (\texttt{man git})?
\end{frame}

\begin{frame}{Tipos de sistemas de controle de versão}
  \begin{columns}[c]
    \column{.3\textwidth}\centering
    \textbf{Local}\\(e.g. GNU RCS)\vspace{1em}
    \begin{tikzpicture}[>=Latex]
      \node[draw, rounded corners=.5ex] (a) at (0, 0) {Versão 1};
      \node[draw, rounded corners=.5ex] (b) at (0, -1) {Versão 2};
      \node[draw, rounded corners=.5ex] (c) at (0, -2) {Versão 3};
      \draw[->] (a) -- (b);
      \draw[->] (b) -- (c);
    \end{tikzpicture}
    \vfill\vspace{1em}
    Banco de dados de arquivos (NoSQL)
    agrupados por versão, sem comunicação remota
    \column{.35\textwidth}\centering
    \textbf{Centralizado}\\(e.g. CVS, SVN)\vspace{1em}
    \tikz\node {Servidor central}
      child {node {A}}
      child {node {B}}
      child {node {C}};%
    \vfill\vspace{1em}
    Um único servidor é a \emph{source of truth},
    os demais nós não possuem toda informação;
    risco de ponto único de falha
    e interrupção de trabalho por falhas na rede
    \column{.35\textwidth}\centering
    \textbf{Distribuído}\\(e.g. \texttt{git}, \texttt{hg})\vspace{1em}
    \vfill
    \begin{tikzpicture}[>=stealth]
      \draw (-.5, -1.5) rectangle (.5, .5);
      \node[draw, rounded corners=.5ex] (v1) at (0, 0) {};
      \node[draw, rounded corners=.5ex] (v2) at (0, -.5) {};
      \node[draw, rounded corners=.5ex] (v3) at (0, -1) {};
      \draw[->] (v1) -- (v2);
      \draw[->] (v2) -- (v3);

      \draw (1, -1.5) rectangle (2, .5);
      \node[draw, rounded corners=.5ex] (w1) at (1.5, 0) {};
      \node[draw, rounded corners=.5ex] (w2) at (1.5, -.5) {};
      \node[draw, rounded corners=.5ex] (w3) at (1.5, -1) {};
      \draw[->] (w1) -- (w2);
      \draw[->] (w2) -- (w3);

      \draw (.25, 1) rectangle (1.25, 3);
      \node[draw, rounded corners=.5ex] (w1) at (.75, 2.5) {};
      \node[draw, rounded corners=.5ex] (w2) at (.75, 2) {};
      \node[draw, rounded corners=.5ex] (w3) at (.75, 1.5) {};
      \draw[->] (w1) -- (w2);
      \draw[->] (w2) -- (w3);

      \draw[<->, dashed, thick, inner sep=1ex] (0, .5) -- (.5, 1);
      \draw[<->, dashed, thick, inner sep=1ex] (1.5, .5) -- (1, 1);
      \draw[<->, dashed, thick, inner sep=1ex] (.5, -1) -- (1, -1);

    \end{tikzpicture}
    \vfill\vspace{1em}
    Toda informação está (ou pode estar) disponível em todos os nós.
  \end{columns}
\end{frame}

\begin{frame}{\emph{Snapshot} VS \emph{delta patching}}
  Uma das escolhas mais relevantes do \emph{design} do \texttt{git}
  é o fato de que todos os arquivos são armazenados integralmente
  (\emph{snapshot}).
  Ou seja, \textbf{todos os arquivos} em todas as suas versões
  \textbf{são armazenados por completo},
  e as ``\textbf{mudanças aplicadas}''
  \textbf{são} na realidade \textbf{computadas} sob demanda.
  \vfill%
  \begin{tabular}{cccccc}
      & Commit1 & Commit2 & Commit3 & Commit4 & Commit5 \\
    \texttt{README}
      & R1
      & \textcolor{gray}{...R1...}
      & \textcolor{gray}{...R1...}
      & \textcolor{gray}{...R1...}
      & R2
      \\
    \texttt{a.py}
      & \textcolor{lightgray}{N/A}
      & m1
      & m2
      & \textcolor{gray}{...m2...}
      & m3
      \\
    \texttt{b.py}
      & \textcolor{lightgray}{N/A}
      & \textcolor{lightgray}{N/A}
      & \textcolor{lightgray}{N/A}
      & p1
      & \textcolor{gray}{...p1...}
      \\
  \end{tabular}
  \vfill%
  Isso permite que todo arquivo seja acessado diretamente,
  sem a necessidade de ``reconstruir'' cada arquivo
  a partir de suas mudanças individuais
  (contraste com CVS e \texttt{darcs}).
\end{frame}

\begin{frame}{Commits no \texttt{git}}
  Podemos interpretar o \texttt{git} como um banco de dados NoSQL
  de ``\emph{snapshots}'' imutáveis, as versões ou \emph{commits}.
  \emph{Commits}:
  \begin{itemize}
    \item
      Possuem um \emph{hash} determinístico (\emph{git} utiliza SHA-1)
    \item
      Consistem em um conjunto de arquivos
      (conteúdo, \emph{path} e \emph{bit} de execução),
      além de metadados: mensagem (título e descrição),
      autor, \emph{timestamp} de autoria,
      \emph{committer}, \emph{timestamp} do \emph{commit},
      lista ordenada de \emph{parents} (ancestrais imediatos)
      referenciados pelos seus \emph{hashes}
    \item
      São referenciáveis, por exemplo através de seu \emph{hash}
      ou seus primeiros 7 ou mais \emph{nibbles} (dígitos hexadecimais)
    \item
      Formam um DAG (grafo direcional acíclico)
  \end{itemize}
\end{frame}

\begin{frame}[fragile]{\emph{Show me the co... commmand lines}}
  \vspace{-.8em}%
  \inputminted{bash}{01_init.sh}
  Para o primeiro uso, é necessário configurar o nome e e-mail:
  \vspace{-.8em}%
  \inputminted{bash}{02_config.sh}
\end{frame}

\begin{frame}{Modelo do git para commits e respectivos comandos}
  \begin{tikzpicture}[
    node distance=4ex, >=stealth', thick,
    LocalNode/.style={draw, rounded corners=.5ex, color=blue!40!black},
    RemoteNode/.style={LocalNode, color=red!40!black, xshift=-3ex},
    VertBar/.style={node distance=7cm},
    MidTxt/.style={above, scale=1, midway},
    MidTxtBelow/.style={MidTxt, below},
    DiffLine/.style={o-Parenthesis, color=green!40!black},
  ]
    \node[LocalNode] (w) {Workspace};
    \node[LocalNode, right=of w] (s) {Staging};
    \node[LocalNode, right=of s] (l) {Repositório local};
    \node[RemoteNode, right=of l] (r) {Repositório remoto};
    \node[below of=w, VertBar] (wg) {};
    \node[below of=s, VertBar] (sg) {};
    \node[below of=l, VertBar] (lg) {};
    \node[below of=r, VertBar] (rg) {};
    \draw[dashed, color=lightgray] (w) -- (wg);
    \draw[dashed, color=lightgray] (s) -- (sg);
    \draw[dashed, color=lightgray] (l) -- (lg);
    \draw[dashed, color=lightgray] (r) -- (rg);
    \draw[->] ($(w)!0.15!(wg)$) --
      node[MidTxt]{add}
      ($(s)!0.15!(sg)$);
    \draw[->] ($(s)!0.18!(sg)$) --
      node[MidTxt]{commit}
      ($(l)!0.18!(lg)$);
    \draw[->] ($(w)!0.21!(wg)$) --
      node[MidTxtBelow]{commit -a}
      ($(l)!0.21!(lg)$);
    \draw[<-] ($(r)!0.33!(rg)$) --
      node[MidTxt]{push}
      ($(l)!0.33!(lg)$);
    \draw[<-] ($(l)!0.36!(lg)$) --
      node[MidTxtBelow]{fetch/pull}
      ($(r)!0.36!(rg)$);
    \draw[->] ($(l)!0.48!(lg)$) --
      node[MidTxt]{reset \emph{ref}}
      ($(s)!0.48!(sg)$);
    \draw[->] ($(s)!0.51!(sg)$) --
      node[MidTxt]{checkout \emph{path}}
      ($(w)!0.51!(wg)$);
    \draw[->] ($(l)!0.54!(lg)$) --
      node[MidTxtBelow]{checkout \emph{ref} / reset -{}-hard}
      ($(w)!0.54!(wg)$);
    \draw[DiffLine] ($(w)!0.78!(wg)$) --
      node[MidTxt]{diff}
      ($(s)!0.78!(sg)$);
    \draw[DiffLine] ($(s)!0.81!(sg)$) --
      node[MidTxt]{diff -{}-staged}
      ($(l)!0.81!(lg)$);
    \draw[DiffLine] ($(w)!0.84!(wg)$) --
      node[MidTxtBelow]{diff \emph{ref}}
      ($(l)!0.84!(lg)$);
  \end{tikzpicture}
  \vfill\vspace{-1em}%
  O \texttt{git diff} não realiza alterações;
  ``\emph{staging}'' também é chamado de \emph{índice} ou \emph{cache};
  \emph{ref} é uma referência a um \emph{commit} (versão).
\end{frame}

\begin{frame}{Referências a \emph{commits}, \emph{branches},
              \emph{detached HEAD}}
  Uma referência especial que merece destaque é a \texttt{HEAD},
  um nome para o commit ``atual'' tomado como base de comparação
  ao avaliar o que mudou no \emph{workspace},
  e que sempre acompanha novos \emph{commits}.
  A \texttt{HEAD} pode apontar para:
  \begin{itemize}
    \item
      Uma \emph{branch} (ramificação) local,
      através de seu \emph{nome}.
      Nesse caso, novos \emph{commits} modificam a \emph{branch},
      continuado-a sequencialmente,
      e a \texttt{HEAD} naturalmente acompanha tais modificações,
      mas continua a apontar para a \emph{branch}.
    \item Um \emph{hash}, referenciando um único \emph{commit}.
      Nesse caso, dizemos que a referência \texttt{HEAD}
      está \texttt{detached} (desprendida).
      Cada novo \emph{commit} altera a \texttt{HEAD},
      mas isso não possui nenhum outro efeito.
  \end{itemize}
  Esse comportamento é característico das \emph{branches} locais;
  todos os outros tipos de referências após um \emph{checkout}
  resultarão em \emph{detached HEAD},
  o que significa que apenas as \emph{branches}
  acompanham naturalmente novos \emph{commits}.
\end{frame}

\begin{frame}[fragile]{Primeiros experimentos com \emph{branches},
                       \emph{tags} e outras \emph{refs}}
  \vspace{-.8em}%
  \inputminted{bash}{03_refs.sh}
\end{frame}

\begin{frame}[fragile]{Exercício de visualização, clone/bisect/grep
                       e Quatérnions!}
  \textbf{Exercício}:
  Quatérnions são números ``4D'' não comutativos
  com propriedades particularmente relevantes
  para modelar rotações em 3D de maneira eficiente e eficaz,
  por exemplo em computação gráfica.
  Encontrar o autor, a data do primeiro commit
  a partir do qual o Sympy passou a ter a palavra \texttt{Quaternion}.
  \inputminted{bash}{04_bisect_exercise.sh}

  Clonando via SSH ou HTTPS faz diferença?
  Qual o resultado de \texttt{git remote -v}?
  As datas de autoria e commit batem?
\end{frame}


\begin{frame}
  \begin{columns}[c]
    \column{.5\textwidth}
    \begin{center}
      {\fontsize{5cm}{2.5cm}\selectfont FIM!}
      \vfill
      \vspace{1em}
      \url{\repo}
    \end{center}
    \vspace{3em}
    Referências:
    \column{.5\textwidth}
    \begin{center}
      \qrcode[height=5cm]{\repo}
    \end{center}
  \end{columns}
  \vfill
  \fontsize{8pt}{7pt}\selectfont
  \begin{itemize}[label=--, leftmargin=0mm]
    \item \url{https://www.gnu.org/philosophy/mcvoy.en.html}
      (Richard Stallman, ``Thank You, Larry McVoy'')
    \item \url{https://www.linuxjournal.com/content/git-origin-story}
      (Zack Brown)
    \item \url{%
        https://pt.slideshare.net/slideshow%
               /20150314-grupysp-projetos-open-source-como-colaborar%
               /45854503\#2}
    \item \url{https://git-scm.com/book}
    \item \url{https://github.com/git/git}
    \item \url{https://github.com/danilobellini/git-tutorial-br/}
  \end{itemize}
\end{frame}


\end{document}
