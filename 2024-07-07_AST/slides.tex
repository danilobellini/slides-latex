\documentclass[utf8]{beamer}

\usepackage[T1]{fontenc}
\usepackage[brazil]{babel}
\usepackage{sourcecodepro}
\usepackage{minted}
\usepackage[absolute, overlay]{textpos}
\usepackage{changepage} % adjustwidth environment
\usepackage{qrcode}
\usepackage{enumitem}
\usepackage{tikz}


\usetikzlibrary{graphs, quotes}

\definecolor{codebgcolor}{HTML}{FFFFFF}
\definecolor{coderulecolor}{HTML}{5F7FFF}
\definecolor{outerbgcolor}{HTML}{FFFFEE}

\setminted{python3, autogobble,
           breaklines, breakanywhere,
           breaksymbolindentleft=0pt, breaksymbolindentright=0pt,
           breaksymbolsepleft=3pt, breaksymbolsepright=3pt,
           breaksymbolright=...,
           bgcolor=codebgcolor, style=borland,
           fontsize=\fontsize{8pt}{8pt},
           frame=lines, rulecolor=coderulecolor, framerule=.7pt}
\setmintedinline{bgcolor={}}

\mode<presentation>
\usetheme{Warsaw}
\usecolortheme{wolverine}
\setbeamercolor{background canvas}{bg=outerbgcolor}
\setbeamercolor{author in head/foot}{bg=black, fg=lightgray}
\beamertemplatenavigationsymbolsempty

\setbeamertemplate{footline}{\leavevmode\hbox{%
  \begin{beamercolorbox}[wd=.31\paperwidth, ht=2.25ex, dp=1ex, center]
                        {author in head/foot}%
    \usebeamerfont{author in head/foot}%
      Danilo J. S. Bellini\hfill\texttt{@danilobellini}%
  \end{beamercolorbox}%
  \begin{beamercolorbox}[wd=.38\paperwidth, ht=2.25ex, dp=1ex, center]
                        {title in head/foot}%
    \usebeamerfont{title in head/foot}%
      \insertshorttitle: \insertshortsubtitle%
  \end{beamercolorbox}%
  \begin{beamercolorbox}[wd=.24\paperwidth, ht=2.25ex, dp=1ex, center]
                        {author in head/foot}%
    \usebeamerfont{author in head/foot}%
      Python Sudeste\hfill%
      \insertshortdate%
  \end{beamercolorbox}%
  \begin{beamercolorbox}[wd=.07\paperwidth, ht=2.25ex, dp=1ex, center]
                        {title in head/foot}%
    \usebeamerfont{title in head/foot}%
      \insertframenumber\,/\,\inserttotalframenumber%
  \end{beamercolorbox}%
}}

\setbeamertemplate{background}{%
  \tikz[overlay,remember picture]%
    \node[opacity=0.05] at (current page.center){%
      \includesvg{.5\paperwidth}{pyse_tags}%
    };%
}

\title{AST}
\subtitle{O que é e como usar o módulo padrão?}
\author{Danilo J. S. Bellini \\ \texttt{@danilobellini}}
\date{2024-07-07}

\setlength{\TPHorizModule}{\paperwidth}
\setlength{\TPVertModule}{\paperheight}

\setitemize{label=\usebeamerfont*{itemize item}%
  \usebeamercolor[fg]{itemize item}%
  \usebeamertemplate{itemize item}%
}

\newcommand{\includesvg}[2]{%
  \ifnum\pdfstrcmp{\pdffilemoddate{#2.svg}}%
                  {\pdffilemoddate{#2.pdf}}>0%
    {\immediate\write18%
     {inkscape -z -D --file=#2.svg --export-pdf=#2.pdf --export-latex}%
    }%
  \fi%
  \centering%
  \resizebox{#1}{!}{%
    \input{#2.pdf_tex}%
  }%
}

\newcommand{\repo}[0]{https://github.com/danilobellini/slides-latex}


\begin{document}


\begin{frame}
  \titlepage
  \center
  {\huge Python Sudeste 2024 @ UFSCar/SP}

  \begin{textblock}{0}(.12, .55)%
    \includesvg{.18\paperwidth}{logo_pyse}%
  \end{textblock}
  \begin{textblock}{0}(.67, .5)%
    \tikz\node {Abstract}
      child {node {Syntax}}
      child {node {Tree}};%
  \end{textblock}
\end{frame}


\begin{frame}{O que acontece com o código Python até a execução?}
  Podemos dividir a conversão do código fonte em 4 passos:
  \begin{itemize}
    \item[1.] Código fonte (uma \emph{string})
    \item[2.]
      Sequência de \emph{tokens} resultantes do analisador léxico
      (\texttt{INDENT}/\texttt{DEDENT}, identificadores, literais,
       operadores, \texttt{NEWLINE}, [soft] keywords, e delimitadores)
    \item[3.] \textbf{Árvore sintática} resultante do parser
    \item[4.] \emph{Bytecode} Python compilado (insumo do interpretador)
  \end{itemize}
\end{frame}


\begin{frame}{Por que chamamos de ``abstrata''?}
  Uma árvore sintática é dita ``abstrata''
  quando ela ``descarta'' alguma informação
  mesmo quando esta faz parte da gramática da linguagem,
  mas que não é relevante de ser armazenada por:
  \begin{itemize}
    \item Ser implícita na estrutura da própria árvore; ou
    \item
      Não possuir valor semântico a ser interpretado, isto é,
      não ter relevância na tradução realizada
      no processo de compilação.
  \end{itemize}
  \vfill
  Por exemplo, parênteses opcionais, comentários, linhas em branco e
  ``\texttt{;}'' separando \emph{statements}
  não trazem nenhum valor para o \emph{bytecode} resultante.
  O ``\texttt{:}'' que denota o início de um novo bloco
  está sempre implícito como parte da sintaxe dos respectivos blocos,
  não é preciso que esse detalhe de sintaxe tenha um ``nó'' próprio
  para representá-lo.
\end{frame}


\begin{frame}[fragile]{Um \emph{Hello world} com AST!}
  \vspace{-.8em}%
  \begin{minted}{pycon}
    >>> import ast
    >>> código = "print('Python Sudeste em São Carlos!')"

    # ast.parse -> Cria o objeto AST a partir do código
    >>> root = ast.parse(código)  # mode="exec" implícito
    >>> type(root)
    <class 'ast.Module'>

    # ast.dump -> Permite "enxergar" a AST
    >>> print(ast.dump(root, indent=2))
    Module(
      body=[
        Expr(
          value=Call(
            func=Name(id='print', ctx=Load()),
            args=[
              Constant(value='Python Sudeste em São Carlos!')],
            keywords=[]))],
      type_ignores=[])

    # compile -> Pode ser usado para converter AST em bytecode
    >>> code_obj = compile(root, filename="<string>", mode="exec")
    >>> type(code_obj)  # Este tipo é o types.CodeType
    <class 'code'>

    # exec/eval -> Permitem rodar o bytecode
    >>> exec(code_obj)
    Python Sudeste em São Carlos!
  \end{minted}
\end{frame}


\begin{frame}[fragile]{``Raiz'' da AST gerada: os 4 modos}
  A gramática abstrata do Python
  define para o elemento raiz \texttt{mod}:
  \vspace{-.8em}%
  \begin{minted}{text}
    mod = Module(stmt* body, type_ignore* type_ignores)
        | Interactive(stmt* body)
        | Expression(expr body)
        | FunctionType(expr* argtypes, expr returns)
  \end{minted}
  \vfill
  \begin{tabular}{rcl}
    \texttt{mode}
      & Tipo devolvido por \mintinline{python}{ast.parse}
      & Conteúdo do \texttt{body} \\ \hline
    \mintinline{python}{"exec"}
      & \mintinline{python}{ast.Module}
      & Lista de nós (\emph{statements}) \\
    \mintinline{python}{"single"}
      & \mintinline{python}{ast.Interactive}
      & Lista de nós (uma linha) \\
    \mintinline{python}{"eval"}
      & \mintinline{python}{ast.Expression}
      & Um único nó (expressão) \\
    \mintinline{python}{"func_type"}
      & \mintinline{python}{ast.FunctionType}
      & Não há (legado)
  \end{tabular}
  \vspace{.5em}\vfill
  Podemos ver o resultado de \mintinline{python}{ast.dump}
  com \mintinline{bash}{python -m ast}:
  \vspace{-.8em}%
  \inputminted{console}{eval_console.txt}
\end{frame}


\begin{frame}{Expressões VS \emph{statements};
              \mintinline{python}{ast.Expression} VS
              \mintinline{python}{ast.Expr}}
  Os nós da AST são objetos que herdam de \mintinline{python}{ast.AST}.
  \vfill
  Uma expressão é algo que em Python produz um valor resultante, mesmo
  que este seja \mintinline{python}{None}.
  (\texttt{expr} na gramática).
  \vfill
  Um \emph{statement} (\texttt{stmt} na gramática)
  é uma operação ``atômica'' no código,
  tal como uma definição de função ou uma atribuição.
  Todo código Python consiste em \emph{statements}.
  \vfill
  O nó de tipo \mintinline{python}{ast.Expression} serve apenas
  como raiz do modo \texttt{eval},
  as expressões são na realidade os valores que podem constar
  em seu campo \texttt{body}.
  \vfill
  Uma expressão pode ser usada como um \emph{statement},
  por exemplo em uma chamada de função.
  O \mintinline{python}{ast.Expr} serve para esse fim,
  contendo em seu campo \texttt{value} a expressão propriamente dita,
  convertendo-a em \texttt{statement}, para a finalidade sintática.
\end{frame}


\begin{frame}[fragile]{O modo \texttt{single}
                       e o legado \texttt{func\_type}}
  O modo \texttt{single}
  unifica o ``EP'' do REPL (\emph{read-eval-print-loop}),
  fazendo chamadas ao \mintinline{python}{sys.displayhook}
  para os nós \mintinline{python}{ast.Expr}
  que constarem no \texttt{body}.
  Isto é, os resultados das ``expressões como \emph{statements}''
  são exibidos como se tudo fosse digitado no terminal do Python
  em uma única linha lógica
  (um ou mais \emph{statements} separados por ``\texttt{;}'').
  \vfill
  O modo \texttt{func\_type} é um legado (Python < 3.5)
  contendo anotações de tipo para funções
  presentes em comentários como

  \mintinline{python}{# type: (int, complex) -> str}.
  \vspace{-.8em}%
  \inputminted{console}{func_type_console.txt}
\end{frame}


\begin{frame}[fragile]{O \emph{built-in} \mintinline{python}{compile}}
  Embora exibido anteriormente para compilar AST para \emph{bytecode},
  esse built-in tradicionalmente recebe como entrada
  código Python como uma \emph{string}.
  Em particular, seu quarto argumento posicional, \texttt{flags}
  pode ser atribuído a \mintinline{python}{ast.PyCF_ONLY_AST},
  nesse caso o comportamento do \mintinline{python}{compile}
  será o mesmo do \mintinline{python}{ast.parse}
  (converter código fonte Python em AST).
  \vfill
  Quando usado para compilar AST para \emph{bytecode}
  O modo fornecido para o \mintinline{python}{compile}
  deve ser o mesmo usado para criar a AST,
  do contrário teremos um \mintinline{python}{TypeError}.
\end{frame}


\begin{frame}[fragile]{Casos de uso para AST}
  \begin{itemize}
    \item Avaliação de características de um código (análise estática)
          e.g. com \mintinline{python}{ast.NodeVisitor}
               ou \mintinline{python}{ast.walk}
      \begin{itemize}
        \item Garantia de segurança/isolamento
              (\mintinline{python}{ast.literal_eval})
        \item Teste de arquitetura
        \item Garantia de equivalência entre códigos (\emph{formatter})
        \item Cálculo de complexidade ciclomática (McCabe)
      \end{itemize}
    \item Transformação de um código
          (e.g. via \mintinline{python}{ast.NodeTransformer})
      \begin{itemize}
        \item
          Remoção de \emph{statements} de um bloco
          (e.g. \mintinline{python}{assert})
        \item
          Coleta de um fragmento de um arquivo,
          (e.g. para o \texttt{setup.py} obter
                o \mintinline{python}{__version__}
                sem importar o módulo/pacote)
        \item Substituição de um fragmento de uma expressão
        \item Inserção de \emph{statements} artificiais
        \item Macro sintática
      \end{itemize}
    \item Geração de código em tempo de execução (AST sintética)
      \begin{itemize}
        \item Metaprogramação
        \item Otimização
      \end{itemize}
  \end{itemize}
\end{frame}


\begin{frame}{Exemplo de síntese: DynagRPC v0.1!}
  Biblioteca criada
  para tornar a implementação de servidores gRPC "menos não-pythonico".
  Código criado para um cliente,
  aberto com autorização para a realização da presente palestra.
  \vfill
  A classe \texttt{GrpcTypeCastRegistry} implementa um registro
  de conversores entre objetos \emph{protobuf}
  e objetos de tipos nativos do Python (principalmente dicionários)
  utilizando síntese de AST.
  \vfill
  \url{https://github.com/danilobellini/dynagrpc}
  \vfill
  Na síntese, a ``posição no código''
  (\texttt{lineno} e \texttt{col\_offset})
  precisa ser preenchida nos nós da AST,
  e há funções como \mintinline{python}{ast.copy_location}
  e \mintinline{python}{ast.increment_lineno}
  feitas para esse fim.
  Esse problema foi resolvido da maneira mais simples,
  com o \mintinline{python}{ast.fix_missing_locations},
  que atribui recursivamente os nós com o mesmo valor do \emph{parent}
  (antecessor imediato, nó ``pai'').
\end{frame}


\begin{frame}
  \begin{columns}[c]
    \column{.5\textwidth}
    \begin{center}
      {\fontsize{5cm}{2.5cm}\selectfont FIM!}
      \vfill
      \vspace{1em}
      \url{\repo}
    \end{center}
    \vspace{3em}
    Referências:
    \column{.5\textwidth}
    \begin{center}
      \qrcode[height=5cm]{\repo}
    \end{center}
  \end{columns}
  \vfill
  \fontsize{8pt}{7pt}\selectfont
  \begin{itemize}[label=--, leftmargin=0mm]
    \item \url{https://docs.python.org/3/library/ast.html}
    \item \url{https://github.com/danilobellini/dynagrpc}
    \item \url{https://greentreesnakes.readthedocs.io}
  \end{itemize}
\end{frame}


\end{document}
