\documentclass[utf8]{beamer}

\usepackage[T1]{fontenc}
\usepackage[brazil]{babel}
\usepackage{inconsolata}
\usepackage{minted}

\definecolor{codebgcolor}{HTML}{FFFFFF}
\definecolor{coderulecolor}{HTML}{5F7FFF}
\definecolor{outerbgcolor}{HTML}{F0F0FF}

\setminted{python3, autogobble,
           breaklines, breakanywhere,
           breaksymbolindentleft=0pt, breaksymbolindentright=0pt,
           breaksymbolsepleft=3pt, breaksymbolsepright=3pt,
           breaksymbolright=...,
           bgcolor=codebgcolor, style=paraiso-light,
           fontsize=\fontsize{8pt}{8pt},
           frame=lines, rulecolor=coderulecolor, framerule=.7pt}
\setmintedinline{bgcolor={}}

\mode<presentation>
\usetheme{Warsaw}
\setbeamercolor{structure}{fg=red!20!black}
\setbeamercolor{background canvas}{bg=outerbgcolor}
\beamertemplatenavigationsymbolsempty

\setbeamertemplate{footline}{\leavevmode\hbox{%
  \begin{beamercolorbox}[wd=.35\paperwidth, ht=2.25ex, dp=1ex, center]
                        {author in head/foot}
    \usebeamerfont{author in head/foot}
      Danilo J. S. Bellini \hfill \texttt{@danilobellini}
  \end{beamercolorbox}%
  \begin{beamercolorbox}[wd=.65\paperwidth, ht=2.25ex, dp=1ex, center]
                        {title in head/foot}
    \usebeamerfont{title in head/foot}
      \insertshorttitle \hfill
      ETEC Uirapuru -- SP \hfill
      \insertshortdate \hfill
      \insertframenumber\,/\,\inserttotalframenumber
  \end{beamercolorbox}%
}}

\title{Segurança da Informação}
\author{Danilo J. S. Bellini \\ \texttt{@danilobellini}}
\date{2018-10-18}


\begin{document}


\begin{frame}
  \titlepage
  \center Semana de informática \\ ETEC Uirapuru -- SP
\end{frame}


\begin{frame}[fragile]{GPG~-- \emph{GNU Privacy Guard}}
  O GPG é uma implementação em software livre (GPLv3)
  que atende ao OpenPGP (maiores detalhes no próximo slide)
  sem utilizar softwares/algoritmos patenteados/restritos/privados.
  \begin{minted}{shell}
    gpg --gen-key # Criar chaves (par público/privado)
    gpg -k        # Visualizar chaves disponíveis

    # Exportando/importanto chaves públicas (para um dado keyID)
    gpg --keyserver pgp.mit.edu --send-keys keyID
    gpg --keyserver pgp.mit.edu --recv-keys keyID
    gpg --export --armor danilo.bellini@gmail.com > my.key
    gpg --import my.key

    # Assinatura
    gpg -b message.txt                       # Assina (cria ".sig")
    gpg --verify message.txt.sig message.txt # Verifica assinatura

    # Codificando/criptografando (crypt) p/ um destinatário específico
    gpg -o encrypted.gpg -r danilo.bellini@gmail.com -e message.txt

    # Decodificando/decriptografando (decrypt) com a chave privada
    gpg -o decrypted.txt -d encrypted.gpg

    # Comandos utilizados na apresentação (além de vim, cat, rm, ...)
    seq 15 > message.txt     # Cria message.txt c/ sequência de 1 a 15
    hexdump -C encrypted.gpg # Visualizando arquivos como binários
  \end{minted}
\end{frame}


\begin{frame}{Referências}
  O material abaixo está disponível apenas em inglês.
  \begin{itemize}
    \item Livro de GPG:
          \url{https://www.gnupg.org/gph/en/manual/book1.html}
    \item Tutorial de GPG:
          \url{https://www.futureboy.us/pgp.html}
    \item RFC1991 (PGP, obsoleto):
          \url{https://tools.ietf.org/html/rfc1991}
    \item RFC4880 (OpenPGP):
          \url{https://tools.ietf.org/html/rfc4880}
  \end{itemize}
\end{frame}


\begin{frame}
  \begin{center}\fontsize{5cm}{2.5cm}\selectfont
    FIM!
  \end{center}
\end{frame}


\end{document}
