\documentclass[utf8]{beamer}

\usepackage[T1]{fontenc}
\usepackage[brazil]{babel}
\usepackage{inconsolata}
\usepackage{minted}
\usepackage{tabu}
\usepackage[absolute,overlay]{textpos}

\definecolor{codebgcolor}{HTML}{FFFFFF}
\definecolor{coderulecolor}{HTML}{5F7FFF}
\definecolor{outerbgcolor}{HTML}{F0FFF0}

\setminted{python3, autogobble,
           breaklines, breakanywhere,
           breaksymbolindentleft=0pt, breaksymbolindentright=0pt,
           breaksymbolsepleft=3pt, breaksymbolsepright=3pt,
           breaksymbolright=...,
           bgcolor=codebgcolor, style=paraiso-light,
           fontsize=\fontsize{8pt}{8pt},
           frame=lines, rulecolor=coderulecolor, framerule=.7pt}
\setmintedinline{bgcolor={}}

\mode<presentation>
\usetheme{Warsaw}
\usecolortheme{seahorse}
\setbeamercolor{structure}{fg=cyan!90!black}
\setbeamercolor{background canvas}{bg=outerbgcolor}
\beamertemplatenavigationsymbolsempty

\setbeamertemplate{footline}{\leavevmode\hbox{%
  \begin{beamercolorbox}[wd=.35\paperwidth, ht=2.25ex, dp=1ex, center]
                        {author in head/foot}
    \usebeamerfont{author in head/foot}
      Danilo J. S. Bellini \hfill \texttt{@danilobellini}
  \end{beamercolorbox}%
  \begin{beamercolorbox}[wd=.65\paperwidth, ht=2.25ex, dp=1ex, center]
                        {title in head/foot}
    \usebeamerfont{title in head/foot}
      \insertshorttitle~--~\insertshortsubtitle \hfill
      \insertshortdate \hfill
      \insertframenumber\,/\,\inserttotalframenumber
  \end{beamercolorbox}%
}}

\title{Sanic 18.12 LTS}
\subtitle{Novos recursos e websockets}
\author{Danilo J. S. Bellini \\ \texttt{@danilobellini}}
\date{2019-03-09}

\setlength{\TPHorizModule}{\paperheight}
\setlength{\TPVertModule}{\paperheight}


\begin{document}


\begin{frame}
  \titlepage
  \center GruPy-ABC @ Fatec Mauá -- SP
  \begin{textblock}{0}(.9, .4)%
    \includegraphics[height=110pt]{../2018-08-25_Sanic/sanic.png}%
  \end{textblock}
  \begin{textblock}{0}(.05, .6)%
    \includegraphics[height=30pt]{sanic-framework-logo-400x97.png}%
  \end{textblock}
\end{frame}


\begin{frame}{Isto é uma continuação!}
  Estes \emph{slides} são uma continuação
  do apresentado na Flask Conf 2018 em 2018-08-25,
  utilizando o Sanic 18.12.0 ao invés do 0.7.0
  (versão estável que estava no PyPI na época).
  \vfill
  Estes slides dizem respeito ao versionamento do projeto,
  às novidades de sua versão 18.12.0,
  e aos uso de WebSockets com o framework.
  Para uma introdução ao Sanic é recomendável
  ver os já citados slides da Flask Conf.
  \vfill
  Todos os \emph{slides} podem ser obtidos em
  \url{https://github.com/danilobellini/slides-latex}.
\end{frame}


\begin{frame}{Versões do Sanic}
  \begin{center}
    \begin{tabu}{cl}
      2016-05-25 & Primeiro commit (\#60f1004) \\
      2016-10-01 & 0.0.1 (\#afeed3c) \\
      2016-10-06 & 0.0.2 (\#74b0cba) \\
      2016-10-14 & 0.1.0 (\#5a7447e) \\
      2016-10-16 & 0.1.2 (Primeiro release com \emph{tag})\\
    \end{tabu}
    \vfill
    \begin{columns}[t]
      \column{.33\textwidth}
      \begin{tabu}{cl}
        2016-10-16 & 0.1.3 \\
        2016-10-18 & 0.1.4 \\
        2016-10-23 & 0.1.5 \\
        2016-10-25 & 0.1.6 \\
        2016-10-25 & 0.1.7 \\
        2016-11-29 & 0.1.8 \\
        2016-12-24 & 0.1.9 \\
        2017-01-14 & 0.2.0 \\
      \end{tabu}
      \column{.33\textwidth}
      \begin{tabu}{cl}
        2017-01-27 & 0.3.0 \\
        2017-02-08 & 0.3.1 \\
        2017-02-25 & 0.4.0 \\
        2017-02-28 & 0.4.1 \\
        2017-04-11 & 0.5.0 \\
        2017-04-14 & 0.5.1 \\
        2017-04-24 & 0.5.2 \\
        2017-05-05 & 0.5.3 \\
      \end{tabu}
      \column{.33\textwidth}
      \begin{tabu}{cl}
        2017-05-08 & 0.5.4 \\
        2017-08-02 & 0.6.0 \\
        2017-12-05 & 0.7.0 \\
        2018-08-17 & 0.8.0 \\
        2018-09-06 & 0.8.1 \\
        2018-09-13 & 0.8.2 \\
        2018-09-13 & 0.8.3 \\
        2018-12-27 & 18.12.0 \\
      \end{tabu}
    \end{columns}
  \end{center}
\end{frame}


\begin{frame}{Novidades da versão 0.8}
  O que mudou
  (além de testes/documentação/\emph{bug fixes}/detalhes)?
  \begin{itemize}
    \item Status passou de \emph{Pre-Alpha} para \emph{Beta}
    \item Auto-reload
    \item Grupos e aninhamentos de blueprints
    \item Rotas com UUID
    \item Novo método \texttt{register\_listener}
          como alternativa ao decorator
    \item Diversas melhorias no tratamento de websockets e streaming
    \item Várias internalidades e melhor cobertura de especificações
          (sanitização, cabeçalho HTTP, etc.)
  \end{itemize}
  \vfill
  A partir da v0.8.3, o repositório foi movido para o projeto
  \url{https://github.com/huge-success}.
\end{frame}


\begin{frame}{Novidades da versão 18.12 LTS}
  O que mudou
  (além de testes/documentação/\emph{bug fixes}/detalhes)?
  \begin{itemize}
    \item Novo sistema de versionamento
          (calendário ao invés de semântico)
    \item Cancelamento de tarefa no \texttt{connection\_lost}
    \item Novo \texttt{stream\_large\_files} no \texttt{Sanic.static}
    \item Mudanças internas relativas
          à identificação do IP e porta de requisições,
          e ao tratamento de erros em arquivos de configuração
    \item Criação de métodos
            \texttt{body\_init},
            \texttt{body\_push} e
            \texttt{body\_finish}
          permitindo a customização da classe \texttt{Request}
    \item Logging ``principal'' em \texttt{sanic.root}
          ao invés do logger raiz,
          \texttt{Handler.log} tornou-se obsoleto
  \end{itemize}
\end{frame}


\begin{frame}[fragile]{UUID -- Identificador único universal}
  As rotas podem ter agora \texttt{uuid} como tipo de dado:
  \inputminted{python}{uuid_route.py}
\end{frame}


\begin{frame}[fragile]{WebSockets}
  Veremos agora um exemplo de WebSockets.
  O próximo exemplo consiste em uma simples sala de bate-papo,
  em que a primeira mensagem enviada por um usuário é seu nome/apelido,
  e as mensagens seguintes são os conteúdos.
  \vfill
  Uma possível estrutura para uso de websockets é:
  \begin{minted}{python}
    @app.websocket("/somewhere")
    async def websocket_route(request, ws):
        try: # Conexão aberta!
            while True:
                msg = await ws.recv() # Mensagem recebida
                await ws.send(msg) # Envio de mensagem
        finally:
            pass # Conexão encerrada!
  \end{minted}
  \vfill
  O objeto \texttt{ws} representa a conexão,
  podendo ser utilizado
  dentro das rotinas de tratamento de outras requisições
  para realizar uma comunicação iniciada pelo servidor.
\end{frame}


\begin{frame}[fragile]{WebSockets -- \texttt{ws\_server.py}}
  \inputminted{python}{ws_server.py}
\end{frame}


\begin{frame}[fragile]{WebSockets -- \texttt{ws\_client.html}}
  \inputminted{html}{ws_client.html}
\end{frame}


\begin{frame}[fragile]{Auto-reload}
  Para desenvolvimento,
  é possível aproveitar do recurso de auto-reload,
  fazendo com que o código seja re-executado caso modificado,
  sem interromper esse monitoramento
  mesmo que o código tenha algum erro.
  Para isso, basta fazer o código do servidor ser um script, com:
  \begin{minted}{python}
    from sanic.websocket import WebSocketProtocol
    # [... Colocar aqui o conteúdo do ws_server.py ...]
    if __name__ == "__main__":
        app.run(host="0.0.0.0", port=8000,
                protocol=WebSocketProtocol,
                auto_reload=True)
  \end{minted}
  \vfill
  Para iniciar o \texttt{ws\_server.py} diretamente
  com essa mesma configuração mas sem auto-reload,
  independente de ter o fragmento acima ou de ser um script:
  \begin{minted}{raw}
    python -m sanic --host 0.0.0.0 --port 8000 ws_server.app
  \end{minted}
\end{frame}


\begin{frame}[fragile]{UNIX Socket}
  É possível usar o Sanic de outras formas, por exemplo:
  \vfill
  \begin{columns}[c]
    \column{.5\textwidth}
    \inputminted{python}{unix_socket_sanic.py}
    \column{.5\textwidth}
    Esse é um exemplo simples com um contador global.
    Para verificar o resultado, pode-se usar:
    \begin{minted}{shell}
      curl -s --unix-socket /tmp/sanic.sock http://localhost/ | xargs
    \end{minted}
    \vfill
    Embora não seja realmente uma novidade,
    a documentação do Sanic foi atualizada
    para explicitar que ele pode ser utilizado nesse tipo de situação.
  \end{columns}
\end{frame}


\begin{frame}[fragile]{Download via streaming de arquivo estático}
  Este código serve um arquivo \texttt{sanic.mp4}:
  \inputminted{python}{stream_video.py}
  \vfill
  A rota \texttt{/video\_not\_chunked}
  pode ser utilizada para enviar
  o arquivo de vídeo inteiro de uma só vez,
  em contraste com o picotamento da rota \texttt{/video}.
  Para verificar o número de "pacotes" TCP enviados,
  pode-se utilizar:
  \begin{minted}{shell}
    sudo tcpdump --interface=any port 8000 and '(tcp-syn|tcp-ack)!=0'
  \end{minted}
\end{frame}


\begin{frame}{Referências}
  \begin{itemize}
    \fontsize{.8em}{0}\selectfont
    \item \url{https://youtu.be/of9gIoQpPmA}~---
          Origem do \texttt{sonic.mp4} utilizado no exemplo
    \item \url{https://github.com/huge-success/sanic}
    \item \url{https://sanic.readthedocs.io}
    \item \url{https://sanicframework.org/}
  \end{itemize}
  \vfill
  \begin{center}\fontsize{2cm}{0}\selectfont
    FIM!
  \end{center}
\end{frame}


\end{document}
