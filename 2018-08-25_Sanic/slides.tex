\documentclass[utf8]{beamer}

\usepackage[T1]{fontenc}
\usepackage[brazil]{babel}
\usepackage{inconsolata}
\usepackage{minted}
\usepackage{tabu}
\usepackage[absolute,overlay]{textpos}

\definecolor{codebgcolor}{HTML}{FFFFFF}
\definecolor{coderulecolor}{HTML}{5F7FFF}
\definecolor{outerbgcolor}{HTML}{F0FFF0}

\setminted{python3, autogobble,
           breaklines, breakanywhere,
           breaksymbolindentleft=0pt, breaksymbolindentright=0pt,
           breaksymbolsepleft=3pt, breaksymbolsepright=3pt,
           breaksymbolright=...,
           bgcolor=codebgcolor, style=paraiso-light,
           fontsize=\fontsize{8pt}{8pt},
           frame=lines, rulecolor=coderulecolor, framerule=.7pt}
\setmintedinline{bgcolor={}}

\mode<presentation>
\usetheme{Warsaw}
\usecolortheme{seahorse}
\setbeamercolor{structure}{fg=cyan!90!black}
\setbeamercolor{background canvas}{bg=outerbgcolor}
\beamertemplatenavigationsymbolsempty

\setbeamertemplate{footline}{\leavevmode\hbox{%
  \begin{beamercolorbox}[wd=.35\paperwidth, ht=2.25ex, dp=1ex, center]
                        {author in head/foot}
    \usebeamerfont{author in head/foot}
      Danilo J. S. Bellini \hfill \texttt{@danilobellini}
  \end{beamercolorbox}%
  \begin{beamercolorbox}[wd=.65\paperwidth, ht=2.25ex, dp=1ex, center]
                        {title in head/foot}
    \usebeamerfont{title in head/foot}
      \insertshorttitle~--~\insertshortsubtitle \hfill
      \insertshortdate \hfill
      \insertframenumber\,/\,\inserttotalframenumber
  \end{beamercolorbox}%
}}

\title{Introdução ao Sanic}
\subtitle{O Flask Assíncrono}
\author{Danilo J. S. Bellini \\ \texttt{@danilobellini}}
\date{2018-08-25}

\setlength{\TPHorizModule}{\paperheight}
\setlength{\TPVertModule}{\paperheight}


\begin{document}


\begin{frame}
  \titlepage
  \center Flask Conf @ Developer Hub -- SP
  \begin{textblock}{0}(.9, .4)%
    \includegraphics[height=110pt]{sanic.png}%
  \end{textblock}
\end{frame}


\begin{frame}{Origem do nome}
  \emph{Sonic: The Hedgehog} é um personagem bastante conhecido
  do mundo dos games...
  \emph{Sanic} é o nome dos seus memes feitos às pressas,
  de qualquer jeito!
  \vfill
  \begin{columns}[t]
    \column{.5\textwidth}\scriptsize
    ``Gotta go fast''
    \includegraphics[height=110pt]{gotta_go_fast.png}
    Referência à música de abertura
    do anime Sonic X na versão estadunidense,
    foi em 2008 associado a um desenho de fã
    presente no (agora extinto) Sonic Central.

    \column{.5\textwidth}\scriptsize
    ``Sanic''
    \includegraphics[height=110pt]{sanic_0nyxheart.png}
    Meme criado em março de 2010 por 0nyxheart
    em um vídeo com o título ``How 2 Draw Sanic Hegehog``
    desenhando o personagem no Microsoft Paint.
  \end{columns}
\end{frame}


\begin{frame}{1\textsuperscript{\b{o}} \emph{commit} em 2016-05-26,
              \emph{post} no Reddit em 2016-10-14}
  \emph{E se nativamente o Flask pudesse ser assíncrono
        e $6$ (seis) vezes mais rápido?}
  \vfill
  Eis o Sanic, para quando você ``gotta go fast''
  (precisa ir rápido)!
  \begin{itemize}
    \item Assíncrono~--- \mintinline{python}{async def}
    \item ``Reluzente''~--- Python 3.5+
    \item Simples~--- rotas como no Flask
    \item Leve~--- não precisa de ferramentas especiais
  \end{itemize}
  \vfill
  Dados fornecidos pelo criador, no Reddit:
  {\fontsize{9pt}{11pt}\selectfont
    \begin{tabu} to \textwidth {ccc}
      Framework & Requisições/segundo & Latência média \\
      \cline{1-3}
      Sanic (Python 3.5 + uvloop) & 30,601 & 3.23ms \\
      Flask (gunicorn + meinheld) & 4,988 & 20.08ms \\
    \end{tabu}
  }
  \vfill
  Link: \url{https://www.reddit.com/r/Python/comments/57i301/}
\end{frame}


\begin{frame}{Citações}
  \begin{itemize}
    \item (2016-10-14)
      \emph{Sanic was created because I love the freedom of Flask,
            but dislike deploying it
            and its lack of native async support.}
      (Channel Cat, criador do Sanic)

    \item (2018-03-26)
      \emph{Sanic: python web server that's written to die fast}
      (Andrew Svetlov, criador do \texttt{aiohttp})

    \item (2018-03-28)
      \emph{I'm a fan of Sanic, keep up the good work.}
      (Phil Jones, autor do Quart)
  \end{itemize}
\end{frame}


\begin{frame}[fragile]{\emph{Hello World!}}
  \begin{columns}[t]
    \column{.5\textwidth}Flask\inputminted{python}{01_flask.py}
    \column{.5\textwidth}Sanic\inputminted{python}{01_sanic.py}
  \end{columns}
  \vfill
  Rodando os servidores (porta $1337$):
  \inputminted{shell}{01_run.sh}
\end{frame}


\begin{frame}[fragile]{Rodando como um script}
  Caso se queira personalizar a execução rodando como um script,
  isto é, rodando algo como
  \mintinline{shell}{python -m 02_xxxxx_script}
  diretamente do Python,
  podemos escrever esse script Python como:
  \vfill
  \begin{columns}[t]
    \column{.5\textwidth}Flask\inputminted{python}{02_flask_script.py}
    \column{.5\textwidth}Sanic\inputminted{python}{02_sanic_script.py}
  \end{columns}
\end{frame}


\begin{frame}[fragile]{JSON I/O (POST): Índice de Massa Corporal}
  \begin{columns}
    \column{.6\textwidth}
    \inputminted{python}{03_flask_json.py}
    \vfill
    \inputminted{python}{03_sanic_json.py}

    \column{.4\textwidth}
    \fontsize{.8em}{1.2em}\selectfont
    A sintaxe p/ selecionar os verbos do HTTP é a mesma:
    o argumento nominado \mintinline{python}{methods}
    com a lista de métodos (apenas GET, por padrão).

    \vspace{1em}
    O primeiro uso do \mintinline{python}{request.json}
    avalia o corpo (\emph{body}/\emph{payload}) da requisição,
    devolvendo \mintinline{python}{None} caso não consiga ler o JSON.
    A diferença é que o Sanic ignora o \texttt{Content-Type} de
    entrada e utiliza uma \mintinline{python}{property}.

    \vspace{1em}
    As respostas possuem o header
    \texttt{Content-Type: application/json}
  \end{columns}
\end{frame}


\begin{frame}[fragile]{JSON I/O (POST)~--- Resultados}
  \inputminted{python}{03_curl.txt}
  \vfill
  \begin{columns}
    \column{.5\textwidth}
    Flask + Gunicorn
    \inputminted{raw}{03_flask_headers.txt}

    \column{.5\textwidth}
    Sanic
    \inputminted{raw}{03_sanic_headers.txt}
    Conexões persistentes, baixa latência!
  \end{columns}
\end{frame}


\begin{frame}[fragile]{Status HTTP diferente de $200$ (OK)}
  Entrada inválida?
  A resposta é ``\texttt{HTTP/1.1 400 Bad Request}''!
  \vfill
  \begin{columns}
    \column{.72\textwidth}
    \inputminted[firstline=4]{python}{04_flask_400.py}
    \inputminted[firstline=4]{python}{04_sanic_400.py}

    \column{.28\textwidth}
    \fontsize{.8em}{1.2em}\selectfont
    As funções do módulo \mintinline{python}{response}
    aceitam dois argumentos nominados:
    \begin{itemize}
      \item \texttt{status}~--- Inteiro com o status HTTP
      \item \texttt{headers}~--- Dicionário com modificações
                                 a serem feitas no cabeçalho HTTP
    \end{itemize}
  \end{columns}
\end{frame}


\begin{frame}[fragile]{Duas rotas no mesmo \emph{handler} c/ Flask}
  Uma rota GET sem \emph{query string}
  (i.e., a entrada está no próprio caminho)
  e uma POST (JSON) tratadas pela mesma função.
  \vfill
  \inputminted{python}{05_flask_biroute.py}
  \vfill
  No Flask, blocos \texttt{<TIPO:NOME>}
  são lidos, convertidos e passados como argumentos nominados.
\end{frame}


\begin{frame}[fragile]{Duas rotas no mesmo \emph{handler} c/ Sanic}
  No Sanic, o tipo \mintinline{python}{float} apenas possui outro nome:
  \mintinline{python}{number}.
  \vfill
  \inputminted{python}{05_sanic_biroute.py}
  \vfill
  No Sanic, blocos \texttt{<NOME:TIPO>} (o inverso do Flask)
  são lidos, convertidos e passados como argumentos nominados.
\end{frame}


\begin{frame}{Tipos permitidos nas rotas}
  \begin{itemize}
    \item Flask~--- \texttt{string}, \texttt{int}, \texttt{float},
                    \texttt{path} e \texttt{uuid}
    \item Sanic~--- \texttt{int}, \texttt{number} e \emph{regexes}
  \end{itemize}
  \vfill
  No Flask o uso de expressões regulares (\emph{regexes})
  poderia ser feito processando o conteúdo de uma variável
  definida no caminho com o tipo \texttt{path},
  ou então customizando um conversor parametrizado
  e inserindo-o em \mintinline{python}{app.url_map.converters}.
  \vfill
  \inputminted{python}{06_regex.py}
\end{frame}


\begin{frame}[fragile]{\emph{Middleware}}
  \begin{columns}
    \column{.77\textwidth}
    \inputminted{python}{07_middlewares.py}

    \column{.23\textwidth}
    \fontsize{.8em}{1.2em}\selectfont
    Similar ao \mintinline{python}{before_request}
    e ao \mintinline{python}{after_request}
    do Flask.

    \vspace{1em}
    Há várias coisas acontecendo:
    \fontsize{.9em}{1em}\selectfont
    \begin{itemize}
      \item \emph{Timestamps} e tempo de processamento
      \item Inserção de $3$ \emph{headers}
      \item Autorização (\emph{bearer token})
      \item Acesso ao caminho da requisição
    \end{itemize}
  \end{columns}
\end{frame}


\begin{frame}[fragile]{\emph{Middleware}~---
                       \emph{Headers} dos resultados}
  \begin{columns}
    \column{.5\textwidth}
    \inputminted{raw}{07_headers_200.txt}
    \column{.5\textwidth}
    \fontsize{.8em}{1.2em}\selectfont
    Os caminhos testados foram:
    \begin{itemize}
      \item \texttt{/} \\ (com \emph{Authorization: TEST})
      \item \texttt{/unknown} \\ (com \emph{Authorization: TEST})
      \item \texttt{/another/path} \\ (apenas \emph{headers} do cURL)
    \end{itemize}
  \end{columns}
  \begin{columns}
    \column{.53\textwidth}
    \inputminted{raw}{07_headers_404.txt}
    \column{.53\textwidth}
    \inputminted{raw}{07_headers_401.txt}
  \end{columns}
\end{frame}


\begin{frame}[fragile]{\emph{Rodando um banco de dados PostgreSQL}}
  \inputminted{shell}{08_database.sh}
  \vfill
  O comando \mintinline{shell}{docker run} acima
  roda o PostgreSQL localmente,
  e o segundo comando define
  a variável de ambiente \texttt{PGSQL\_URL}
  com o DSN (\emph{data source name}),
  permitindo que a conexão com o banco
  seja realizada por meio dessa variável
  ao invés de um valor \emph{hardcoded} no código Python.
\end{frame}


\begin{frame}[fragile]{\emph{Listeners}~---
                       \emph{Timestamp} do banco de dados...
                       \emph{Await}!!!}
  \begin{columns}
    \column{.73\textwidth}
    \inputminted{python}{08_listeners.py}

    \column{.33\textwidth}
    \fontsize{.8em}{1.2em}\selectfont
    Há $4$ listeners possíveis,
    e todos possuem os mesmos parâmetros
    de entrada:
    \begin{itemize}
      \item \mintinline{python}{before_server_start}
      \item \mintinline{python}{after_server_start}
      \item \mintinline{python}{before_server_stop}
      \item \mintinline{python}{after_server_stop}
    \end{itemize}

    \vspace{1em}
    Finalmente assíncrono!
  \end{columns}
  \vfill
  \inputminted{shell}{08_curl.txt}
\end{frame}


\begin{frame}[fragile]{Cuidados com o código síncrono}
  \begin{columns}
    \column{.65\textwidth}
    \inputminted{python}{09_sync.py}
    \column{.34\textwidth}
    \inputminted{raw}{09_sync_result.txt}
  \end{columns}
  \vfill
  \fontsize{.8em}{1em}\selectfont
  Com $1$ worker, $5$ processos realizando requisições simultâneas,
  e um total de $10$ requisições.
  O comando abaixo foi utilizado para realizar as execuções:
  \inputminted{shell}{09_run.sh}
\end{frame}


\begin{frame}[fragile]{\texttt{asyncio.sleep}}
  \begin{columns}
    \column{.72\textwidth}
    \inputminted{python}{09_async.py}
    \column{.34\textwidth}
    \inputminted{raw}{09_async_result.txt}
  \end{columns}
  \vfill
  \fontsize{.8em}{1em}\selectfont
  Que tal $5$ vezes mais rápido?
  Não use o \mintinline{python}{time.sleep}
  em código assíncrono,
  pois ele é bloqueante!
\end{frame}


\begin{frame}[fragile]{\texttt{concurrent.futures.ThreadPoolExecutor}}
  \begin{columns}
    \column{.74\textwidth}
    \inputminted{python}{09_executor.py}
    \column{.34\textwidth}
    \inputminted{raw}{09_executor_result.txt}
  \end{columns}
  \vfill
  \fontsize{.8em}{1em}\selectfont
  Alternativa híbrida!
  Por que não usar um pool de threads para rodar o código bloqueante
  enquanto o restante continua assíncrono?
  \vfill
  O \mintinline{python}{None} desse excerto
  denota o \mintinline{python}{ThreadPoolExecutor}.
  \vfill
  O \mintinline{python}{app.loop} é o \emph{loop} do \texttt{uvloop},
  que também pode ser obtido através do
  \mintinline{python}{asyncio.get_event_loop()}.
\end{frame}


\begin{frame}[fragile]{\emph{Blueprints}~---
                       Rotas (\emph{request handlers})}
  \fontsize{.8em}{1em}\selectfont
  Se trocarmos a classe \mintinline{python}{Sanic}
  pela classe \mintinline{python}{Blueprint},
  a princípio nada muda...
  \vfill
  \begin{center}Módulo \texttt{imc10.py}\end{center}
  \inputminted{python}{imc10.py}
  \vfill
  \emph{Blueprints} podem fazer parte da aplicação,
  trata-se de uma forma de ``picotar'' em diferentes módulos
  sem o problema de importação cíclica,
  dado que o objeto \mintinline{python}{app}
  é tipicamente criado no contexto do módulo principal.
\end{frame}

\begin{frame}[fragile]{\emph{Blueprints}~--- \emph{Middleware}}
  \fontsize{.8em}{1em}\selectfont
  A rigor, mesmo \emph{middlewares} e \emph{listeners}
  podem fazer parte do blueprint, porém eles serão sempre globais
  (isto é, o \emph{middleware} não é acionado apenas com as rotas
   do módulo com o blueprint em que ele se encontra, mas com todas
   as rotas tratadas pela aplicação).
  \vfill
  \begin{center}Módulo \texttt{middle10.py}\end{center}
  \inputminted{python}{middle10.py}
\end{frame}

\begin{frame}[fragile]{Arquivos estáticos e \emph{blueprints}}
  Bora unificar, aproveitando para fornecer arquivos estáticos!
  \vfill
  \inputminted{python}{10_bp.py}
  \vfill
  Há ainda diversos outros recursos
    (e.g. prefixo de caminho,
          uso de nomes e do \mintinline{python}{url_for}
            para obtenção dos caminhos,
          etc.).
\end{frame}


\begin{frame}[fragile]{\emph{Fallback} de exceções}
  Nos exemplos em que haviam exceções não tratadas,
  o servidor devolve um erro interno do servidor
  (status HTTP $500$) com uma mensagem em HTML.
  Algo similar ocorreu com os erros $404$ (não encontrado),
  $405$ (método não permitido) e $401$ (não autorizado)
  nos exemplos anteriores.
  Essas exceções poderiam ser, todas, filtradas para terem
  suas mensagens substituídas por mensagens em JSON.
  Para isso, basta usar o decorador \texttt{exception}
  (na aplicação ou em um \emph{blueprint}).
  Por exemplo:
  \inputminted{python}{exc10.py}
\end{frame}


\begin{frame}[fragile]{Respostas sempre JSON!}
  Aplicando o novo \emph{blueprint},
  \inputminted{python}{10_json_only.py}
  \vfill
  As requisições para rotas inválidas
  agora resultam em JSON válido como resposta,
  acompanhando o cabeçalho HTTP:
  \inputminted{raw}{10_responses.txt}
\end{frame}


\begin{frame}{E muitos outros recursos!}
  \begin{itemize}
    \item Websockets (suporte nativo);
    \item Testes automatizados com
      \mintinline{python}
                 {request, response = app.test_client.get("/rota")}
      (ou com o \emph{plugin} \texttt{pytest-sanic})
    \item \emph{Class-Based Views},
      com \mintinline{python}{sanic.views.HTTPMethodView}
      e \mintinline{python}{app.add_route};
    \item Configuração (\mintinline{python}{app.config});
    \item SSL;
    \item Compartilhamento de loop com outros recursos assíncronos;
    \item
      \href{https://github.com/channelcat/sanic/wiki/Examples}{%
        \nolinkurl{Exemplos}%
      } na wiki do Sanic (GitHub)
      com \texttt{aiohttp} (cliente HTTP assíncrono),
      Jinja2, aiopeewee, aiopg, Motor, etc.;
    \item
      \href{https://github.com/channelcat/sanic/wiki/Extensions}{%
        \nolinkurl{Plugins}%
      } (ou extensões) listadas na wiki do Sanic.
  \end{itemize}
\end{frame}


\begin{frame}{Futuro do Sanic}
  A versão mais recente do Sanic no PyPI é a \texttt{0.7.0},
  embora a \emph{tag} \texttt{0.8.0} já esteja no repositório.
  \vfill
  O status ainda é \emph{Pre-Alpha},
  mas o desenvolvimento está bastante rápido!
  \vfill
  Uma seleção de issues em andamento:
  \begin{itemize}
    \item
      \href{https://github.com/channelcat/sanic/issues/1176}{%
        \nolinkurl{issue 1176}%
      }~---
      O modo de \emph{streaming} está incompleto;

    \item
      \href{https://github.com/channelcat/sanic/issues/1265}{%
        \nolinkurl{issue 1265}%
      }~---
      Existe a intenção de prover compatibilidade com o ASGI
      (\emph{Asynchronous Server Gateway Interface});
  \end{itemize}
  \vfill
  Será que saberemos quem é ``Channel Cat'', o autor do Sanic?
\end{frame}


\begin{frame}{Referências}
  \begin{itemize}
    \fontsize{.8em}{0}\selectfont
    \item \url{https://youtu.be/VTHsOSGJHN0}~--- Abertura de Sonic X
    \item \url{https://youtu.be/3f_cAy8ugvQ}~--- História do meme
    \item \url{https://youtu.be/of9gIoQpPmA}~--- Reupload do 0nyxheart
    \item \url{https://knowyourmeme.com/memes/sanic-hegehog}
    \item \url{https://knowyourmeme.com/memes/gotta-go-fast}
    \item \url{https://magic.io/blog/uvloop-blazing-fast-python-networking/}
    \item \url{https://github.com/channelcat/sanic}
    \item \url{https://sanic.readthedocs.io}
    \item \url{https://www.reddit.com/r/Python/comments/57i301/}
  \end{itemize}
  \vfill
  \begin{center}\fontsize{2cm}{0}\selectfont
    FIM!
  \end{center}
\end{frame}


\end{document}
