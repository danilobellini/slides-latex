\documentclass[utf8]{beamer}

\usepackage[T1]{fontenc}
\usepackage[brazil]{babel}
\usepackage{inconsolata}
\usepackage{minted}
\usepackage[absolute, overlay]{textpos}
\usepackage{changepage} % adjustwidth environment
\usepackage{qrcode}
\usepackage{enumitem}
\usepackage{tikz}


\definecolor{codebgcolor}{HTML}{FFFFFF}
\definecolor{coderulecolor}{HTML}{5F7FFF}
\definecolor{outerbgcolor}{HTML}{F7FFF0}

\definecolor{bgbluish}{HTML}{000033}
\definecolor{rulegreen}{HTML}{335533}
\definecolor{outerbgcolor}{HTML}{F7FFF0}

\setminted{python3, autogobble,
           breaklines, breakanywhere,
           breaksymbolindentleft=0pt, breaksymbolindentright=0pt,
           breaksymbolsepleft=3pt, breaksymbolsepright=3pt,
           breaksymbolright=...,
           bgcolor=codebgcolor, style=paraiso-light,
           fontsize=\fontsize{10pt}{10pt},
           frame=lines, rulecolor=coderulecolor, framerule=.7pt}
\setmintedinline{bgcolor={}}

\mode<presentation>
\usetheme{Warsaw}
\usecolortheme{seahorse}
\setbeamercolor{background canvas}{bg=outerbgcolor}
\beamertemplatenavigationsymbolsempty

\setbeamertemplate{footline}{\leavevmode\hbox{%
  \begin{beamercolorbox}[wd=.29\paperwidth, ht=2.25ex, dp=1ex, center]
                        {author in head/foot}%
    \usebeamerfont{author in head/foot}%
      Danilo J. S. Bellini\hfill\texttt{@danilobellini}%
  \end{beamercolorbox}%
  \begin{beamercolorbox}[wd=.38\paperwidth, ht=2.25ex, dp=1ex, center]
                        {title in head/foot}%
    \usebeamerfont{title in head/foot}%
      \insertshorttitle%
  \end{beamercolorbox}%
  \begin{beamercolorbox}[wd=.26\paperwidth, ht=2.25ex, dp=1ex, center]
                        {author in head/foot}%
    \usebeamerfont{author in head/foot}%
      Just Python 3.0\hfill%
      \insertshortdate%
  \end{beamercolorbox}%
  \begin{beamercolorbox}[wd=.07\paperwidth, ht=2.25ex, dp=1ex, center]
                        {title in head/foot}%
    \usebeamerfont{title in head/foot}%
      \insertframenumber\,/\,\inserttotalframenumber%
  \end{beamercolorbox}%
}}

\title{\emph{Assignment expressions} do zero ao anti-herói}
\subtitle{O novo ``walrus'' do Python 3.8 -- PEP 572}
\author{Danilo J. S. Bellini \\ \texttt{@danilobellini}}
\date{2019-11-09}

\setlength{\TPHorizModule}{\paperwidth}
\setlength{\TPVertModule}{\paperheight}

\setitemize{label=\usebeamerfont*{itemize item}%
  \usebeamercolor[fg]{itemize item}%
  \usebeamertemplate{itemize item}%
}

\newcommand{\repo}[0]{https://github.com/danilobellini/slides-latex}


\begin{document}


\begin{frame}
  \titlepage
  \center
  {\huge Just Python 3.0 @ Creditas -- SP}

  \begin{textblock}{0}(.12, .55)%
    \fontsize{5cm}{5cm}\selectfont :=%
  \end{textblock}
  \begin{textblock}{0}(.75, .55)%
    \fontsize{5cm}{5cm}\selectfont :=%
  \end{textblock}
\end{frame}


\begin{frame}{\emph{Walrus} --
              O operador das \emph{assignment expressions} (PEP 572)}
  \centering
  Veremos sobre o operador ``\mintinline{python}{:=}'',
  conhecido como \emph{Walrus} (Morsa)
  \vfill
  \includegraphics[width=.6\paperwidth]{Noaa-walrus30.jpg}
  \vfill
  \includegraphics[width=.25\paperwidth]{rotwalrus.jpg}
  \hfill
  \begin{tikzpicture}
    \node[anchor=north west, inner sep=0] (walrus)
      {\includegraphics[width=.25\paperwidth]{rotwalrus.jpg}};
    \fill[draw=none, fill=outerbgcolor, fill opacity=.4]
      (walrus.north west) -- (walrus.north east) --
      (walrus.south east) -- (walrus.south west) --
      (walrus.north west) -- cycle;
    \fill[draw=none, fill=black, fill opacity=.4]
      (.55, -.65) circle (.1)
      (.55, -1.65) circle (.1);
    \fill[draw=none, fill=black, fill opacity=.4, rounded corners]
      (1.3, -1.6) rectangle (3.1, -1.4)
      (1.3, -.8) rectangle (3.1, -.6);
  \end{tikzpicture}
  \hfill
  \begin{tikzpicture}
    \node[anchor=north west, inner sep=0] (walrus)
      {\includegraphics[width=.25\paperwidth]{rotwalrus.jpg}};
    \fill[draw=none, fill=outerbgcolor, fill opacity=.8]
      (walrus.north west) -- (walrus.north east) --
      (walrus.south east) -- (walrus.south west) --
      (walrus.north west) -- cycle;
    \fill[draw=none, fill=black, fill opacity=.8]
      (.55, -.65) circle (.1)
      (.55, -1.65) circle (.1);
    \fill[draw=none, fill=black, fill opacity=.8, rounded corners]
      (1.3, -1.6) rectangle (3.1, -1.4)
      (1.3, -.8) rectangle (3.1, -.6);
  \end{tikzpicture}
\end{frame}


\begin{frame}[fragile]{O que faz esse operador?}
  Esse operador faz com que uma atribuição possa ser realizada
  em meio a uma expressão,
  não mais apenas como \emph{statement}.
  \vfill
  \begin{minted}{python}

    >>> import re
    >>> msg = "Just Python 3.0 terá lightnings + 12 palestras!"

    # Vamos encontrar um número de pelo menos 2 dígitos em msg
    >>> match = re.search(r"\d{2,}", msg)
    >>> if match:
    ...     print(match.group())
    12

    # A tentação: duplicar código...
    >>> if re.search(r"\d{2,}", msg):
    ...     print(re.search(r"\d{2,}", msg).group())
    12

    # Com assignment expression
    >>> if match := re.search(r"\d{2,}", msg):
    ...     print(match.group())
    12

  \end{minted}
\end{frame}


\begin{frame}[fragile]{Evitando atribuições desnecessárias}
  Supondo que temos \mintinline{python}{p1} e \mintinline{python}{p2}
  como \emph{regexes}, por exemplo
  \mintinline{python}{re.compile(r"^Just\s?Python\s?\S+")}
  e \mintinline{python}{re.compile(r"\S{5,}")}.
  \vspace{.5em}
  \vfill
  \begin{columns}[t]
    \column{.33\textwidth}
    \emph{Greedy/eager}
    \vspace{-1em}
    \begin{minted}{python}
      m1 = p1.search(msg)
      m2 = p2.search(msg)
      if m1:
          r = m1.group()
      elif m2:
          r = m2.group()
      else:
          r = "fallback"
    \end{minted}

    \column{.39\textwidth}
    Só o necessário
    \vspace{-1em}
    \begin{minted}{python}
      m1 = p1.search(msg)
      if m1:
          r = m1.group()
      else:
          m2 = p2.search(msg)
          if m2:
              r = m2.group()
          else:
              r = "fallback"
    \end{minted}

    \column{.44\textwidth}
    Com o \emph{walrus}
    \vspace{-1em}
    \begin{minted}{python}
      if m1 := p1.search(msg):
          r = m1.group()
      elif m2 := p2.search(msg):
          r = m2.group()
      else:
          r = "fallback"
    \end{minted}
  \end{columns}
  \vfill
  Quando queremos fazer várias verificações consecutivas
  evitando realizar atribuições desnecessariamente,
  tínhamos que alterar a estrutura do código
  ou usar algum tipo de quebra, como um \mintinline{python}{return},
  o que também poderia significar uma mudança na estrutura do código.
  As \emph{assignment expressions} permitem o mesmo resultado
  sem gambiarras.
\end{frame}


\begin{frame}[fragile]{Evitando excesso de aninhamentos /
                       quebra da ``linearidade''}
  \vspace{-1em}
  \begin{minted}{python}
    reductor = dispatch_table.get(cls)
    if reductor:
        rv = reductor(x)
    else:
        reductor = getattr(x, "__reduce_ex__", None)
        if reductor:
            rv = reductor(4)
        else:
            reductor = getattr(x, "__reduce__", None)
            if reductor:
                rv = reductor()
            else:
                raise Error("un(shallow)copyable object")

    # O mesmo (baseado na stdlib "copy"), com o walrus
    if reductor := dispatch_table.get(cls):
        rv = reductor(x)
    elif reductor := getattr(x, "__reduce_ex__", None):
        rv = reductor(4)
    elif reductor := getattr(x, "__reduce__", None):
        rv = reductor()
    else:
        raise Error("un(shallow)copyable object")
  \end{minted}
\end{frame}


\begin{frame}[fragile]{Exemplos de casos de uso}
  \vspace{-1em}
  \begin{minted}{python}
    # Utilização de um match de uma expressão regular
    if (match := pattern.search(data)) is not None:
        # Fazer algo com o match

    # Laço que não tem como ser trivialmente escrito de uma vez
    while chunk := file.read(8192):
        process(chunk)

    # Reuso de um valor caro de ser computado
    # (e.g. dentro de um lambda ou de outra expressão)
    [y := f(x), y**2, y**3]

    # Compartilhamento de subexpressão do filtro da comprehension
    # (isto será melhor detalhado adiante)
    filtered_data = [y for x in data if (y := f(x)) is not None]
  \end{minted}
  \vfill
  \vspace{-1em}
  \begin{columns}
    \column{.5\textwidth}
    \begin{minted}{python}
      # Segundo exemplo sem walrus
      chunk = file.read(8192):
      while chunk:
          process(chunk)
          chunk = file.read(8192)
    \end{minted}

    \column{.5\textwidth}
    \begin{minted}{python}
      while True:
          chunk = file.read(8192):
          if not chunk:
              break
          process(chunk)
    \end{minted}
  \end{columns}
\end{frame}


\begin{frame}[fragile]{Recomendações de ``estilo''}
  A PEP 572 traz duas ``recomendações de estilo'',
  situações em que o uso de \emph{statements}
  é indicado ao invés de \emph{assignment expressions}:
  \vfill
  \begin{itemize}
    \item Quando ``tanto faz''
    \item Quando houver ambiguidade relativa à ordem de execução
  \end{itemize}
  \vfill
  No apêndice A da PEP 572,
  há diversos exemplos de código
  cuja legibilidade foi considerada ``melhor'' ou ``pior''
  por Tim Peters.
  Exemplos de situações distintas:
  \begin{minted}{python}
    # Confuso! O "total" aparece muitas vezes em um lugar só
    while total != (total := total + term):
        term *= mx2 / (i*(i+1))
        i += 2
    return total

    # Ok! Mais legível que aninhamentos artificiais em statements
    if (diff := x - x_base) and (g := gcd(diff, n)) > 1:
      return g
  \end{minted}
\end{frame}


\begin{frame}[fragile]{Exemplo final do apêndice A da PEP 572
                       (adaptado)}
  Este é um algoritmo para estimar $\lfloor x^n \rfloor$,
  iniciando por um palpite ``$a$'' maior que o resultado,
  usando apenas inteiros positivos.
  \vfill
  \begin{minted}{python}

    >>> x, n, a = 28, 3, 1000
    >>> while a > (d := x // a ** (n - 1)):
    ...     a = ((n - 1) * a + d) // n
    >>> a
    3

    >>> a = 1000
    >>> while True:
    ...     d = x // a ** (n - 1)
    ...     if a <= d:
    ...         break
    ...     a = ((n - 1) * a + d) // n
    >>> a
    3

  \end{minted}
  \vfill
  Qual implementação é mais legível?
\end{frame}


\begin{frame}[fragile]{Uso em \emph{list/set/dict comprehensions}
                       e \emph{generator expressions}}
  O \emph{walrus} pode aparecer
  em \emph{quase} qualquer lugar em que uma expressão é permitida,
  embora ele tenha de estar com parênteses em algumas situações.
  É importante saber a ordem de execução!
  \vfill
  \vspace{-.5em}
  \begin{minted}{python}

    >>> data = ["  a", "lgumas ", "  ", " strings  ", " ", "aqui"]

    >>> # Ruim! O ".strip()" é calculado duas vezes
    >>> [value.strip() for value in data if value.strip()]
    ['a', 'lgumas', 'strings', 'aqui']

    >>> # Muito melhor!
    >>> [trimmed for value in data if (trimmed := value.strip())]
    ['a', 'lgumas', 'strings', 'aqui']

    >>> # Em dicionários, a chave é avaliada antes do valor
    >>> {(t := value.strip()): len(t) for value in data}
    {'a': 1, 'lgumas': 6, '': 0, 'strings': 7, 'aqui': 4}

  \end{minted}
  \vfill
  \vspace{-.5em}
  No iterável (a parte imediatamente após o \mintinline{python}{in}),
  é proibido o uso de \emph{assignment expressions}
  (\texttt{SyntaxError}:
   \emph{assignment expression cannot be used
         in a comprehension iterable expression}).
\end{frame}


\begin{frame}[fragile]{\emph{Scan} (função de ordem superior) /
                       \emph{accumulate}}
  Esse operador faz com que seja possível implementar
  uma \emph{comprehension} que acessa o valor da iteração anterior
  (um ``acumulador''),
  possibilitando que o \emph{scan} da programação funcional
  seja implementado por meio de uma \emph{list comprehension}.
  \vspace{-.5em}
  \begin{minted}{python}

    >>> # Soma acumulada
    >>> acc = 0
    >>> [acc := acc + v for v in [1, 2, 3, 4, -2, 1, 1]]
    [1, 3, 6, 10, 8, 9, 10]
    >>> v  # O escopo é somente o interno à list comprehension
    Traceback (most recent call last):
      ...
    NameError: name 'v' is not defined

    >>> acc  # Mas, para o walrus, o menor escopo possível
    ...      # é o externo à list comprehension
    10

    >>> # Fibonacci (scan e map em uma só list comprehension)
    >>> mem = 0, 1
    >>> [(mem := (mem[1], sum(mem)))[0] for unused in range(14)]
    [1, 1, 2, 3, 5, 8, 13, 21, 34, 55, 89, 144, 233, 377]

  \end{minted}
\end{frame}


\begin{frame}[fragile]{Escopo e casos não inicializados!}
  \vspace{-1em}
  \begin{minted}{python}

    # O nome precisa ser definido de antemão para ser utilizável
    >>> [undef := None if undef else True for unused in range(9)]
    Traceback (most recent call last):
      ...
    NameError: name 'undef' is not defined

    >>> # O escopo padrão é local (o menor)
    >>> unused = [undef := x for x in range(3) if (tr00 := True)]
    >>> undef, tr00
    (2, True)

    >>> # Mas é mantido o escopo original do nome, se for outro
    >>> def some_func():
    ...     global new_name  # O mesmo vale para nonlocal
    ...     [new_name := x for x in range(19) if (other := True)]
    >>> some_func()
    >>> new_name
    18
    >>> other
    Traceback (most recent call last):
      ...
    NameError: name 'other' is not defined

  \end{minted}
\end{frame}


\begin{frame}[fragile]{Escopo e \texttt{lambda}}
  Em todos os casos, inclusive do \mintinline{python}{lambda},
  o \emph{walrus} honra o escopo,
  e não há nenhuma mudança nas regras de escopo do Python
  por conta do novo operador.
  \vfill
  \begin{minted}{python}

    # O lambda define um escopo próprio!
    >>> p = r"\d+(.*)"
    >>> (lambda d: (m := re.match(p, d)) and m.group(1)
    ... )("123abc")
    'abc'
    >>> m
    Traceback (most recent call last):
      ...
    NameError: name 'm' is not defined

  \end{minted}
  \vfill
  A \textbf{única} exceção é o caso já visto
  das \emph{list/set/dict comprehensions}
  e \emph{generator expressions},
  as quais o escopo mínimo é o imediatamente externo a elas.
\end{frame}


\begin{frame}[fragile]{Sintaxe -- Onde posso usar?}
  A atribuição só pode ser feita para um \textbf{nome},
  e a \emph{assignment expression}
  também pode ser chamada de \emph{named expression}.
  \vspace{-.5em}
  \begin{minted}{python}

    # Cuidado! Para o :=, as vírgulas separam as atribuições!
    >>> a, b = 7, 0   # Imensa diferença entre o = e o :=
    >>> [(a, b := b + a, b - a) for unused in range(4)]
    [(7, 7, 0), (7, 14, 7), (7, 21, 14), (7, 28, 21)]

    >>> [((a, b) := (b + a, b - a)) for unused in range(4)]
    Traceback (most recent call last):
      ...
    SyntaxError: cannot use named assignment with tuple

    # Não pode o nome usado no "for" de uma list comprehension
    >>> [i := i for i in range(10)]
    Traceback (most recent call last):
      ...
    SyntaxError: assignment expression cannot rebind comprehension iteration variable 'i'

  \end{minted}
  \vfill
  \vspace{-.5em}
  Em muitos casos, é necessário o uso de parênteses,
  por exemplo quando se quer atribuir uma \mintinline{python}{tuple}
  a um dado nome.
\end{frame}


\begin{frame}[fragile]{Além das \emph{tuplas}, quando usar parênteses?}
  \vspace{-1em}
  \begin{minted}{python}
    # Quando for uma statement
    y := f(x)    # INVÁLIDO

    # Quando estiver no mesmo "nível" de outra atribuição
    y0 = y1 := f(x)             # INVÁLIDO
    foo(x = y := f(x))          # INVÁLIDO
    def foo(answer = p := 42):  # INVÁLIDO
    y = (a := b := c)           # INVÁLIDO

    # No mesmo nível de um ":", como em annotations e lambdas
    {x := y + 2 : y for y in range(3)}  # INVÁLIDO
    (lambda: x := 1)                    # INVÁLIDO
    def foo(answer: p := 42 = 5):       # INVÁLIDO
    def foo(answer: (p := 42) = 5):     # Válido
  \end{minted}
  \vfill
  Essa lista não é exaustiva,
  há outros lugares em que é necessário colocar parênteses,
  como no \mintinline{python}{if} de uma \emph{list comprehension}
  e dentro de \emph{f-strings}.
  Em todos esses exemplos, colocar parênteses torna a sintaxe válida.
\end{frame}


\begin{frame}[fragile]{Desafios}
  \begin{minted}{python}
    # O que isto faz?
    if any(len(longline := line) >= 100 for line in lines):
        print("Extremely long line:", longline)

    # Qual a diferença entre essas duas listas?
    # Se f = lambda k: max(k ** 2, 1), quais são os resultados?
    [(lambda y: [y, x / y])(f(x)) for x in range(5)]
    [[y := f(x), x / y] for x in range(6)]

    # Qual o valor de x em cada um desses casos?
    x = (x := 2) + 1
    x = (x := ["just"]) + (x := ["python"])
    z = (x := ["just"]) + (x := ["python"])

  \end{minted}
\end{frame}


\begin{frame}{Cadê o anti-herói?}
  O anti-herói \emph{não} é esse novo operador \emph{walrus}
  e nem mesmo algum potencial uso dele
  de maneira despreocupada com a legibilidade.
  O real anti-herói é a \textbf{negatividade}
  de certas interações sociais!
  \vfill
  No dia 2018-07-12,
  GvR enviou uma mensagem na lista \texttt{python-committers}
  c/ o título \emph{Transfer of power}, começando com:
  \vfill
  \begin{adjustwidth}{2em}{2em}
    ``\emph{Now that PEP 572 is done,
    I don't ever want to have to fight so hard for a
    PEP and find that so many people despise my decisions..

    I would like to remove myself entirely from the decision process.
    I'll still be there for a while as an ordinary core dev,
    and I'll still be available to mentor people --
    possibly more available.
    But I'm basically giving myself
    a permanent vacation from being BDFL,
    and you all will be on your own.}''
  \end{adjustwidth}
\end{frame}

\begin{frame}{História (Parte 1/2)}
  \begin{itemize}
    \item (2007-05-07) \emph{Issue} 1714448 propõe
                       \mintinline{python}{if expr as name:}
    \item (2009-03-14) Discussão é iniciada na \texttt{python-ideas},
                       criação da PEP 379
    \item (2009-03-15) ``\emph{a solid -1 from me}'' (GvR),
    \item (2018-02-27) Definição da PEP 572 por Chris Angelico,
                       com o título
                       \emph{Syntax for Statement-Local Name Bindings},
                       e início de novas discussões
                       na \texttt{python-ideas}
                       a partir de uma prova de conceito
                       que ele próprio criou
    \item (2018-02-28) Data oficial da efetiva criação da PEP 572
    \item (2018-03-23)
    Christoph Groth  propõe o uso do \texttt{:=}como operador; e
    ``\emph{I also think it's fair to at least reconsider
            adding inline assignment,
            with the "traditional" semantics
            (possibly with mandatory parentheses).
            This would be easier to learn and understand
            for people who are familiar with it
            from other languages (C++, Java, JavaScript).}'' (GvR)
  \end{itemize}
\end{frame}

\begin{frame}{História (Parte 1/2)}
  \begin{itemize}
    \item (2018-04-11) O título da PEP 572 passa a ser
                       \emph{Assignment Expressions}
    \item (2018-04-15) ``\emph{I strongly prefer \texttt{:=}
                               as inline assignment operator} (GvR)
    \item (2018-05-08) GvR define as regras de escopo
                       relativas ao uso do novo operador
                       em \emph{comprehensions}
    \item (2018-07-11) GvR aprova a PEP 572
    \item (2018-07-12) GvR abandona o cargo de BDFL
  \end{itemize}
\end{frame}


\begin{frame}
  \begin{columns}[c]
    \column{.5\textwidth}
    \begin{center}
      {\fontsize{5cm}{2.5cm}\selectfont FIM!}
      \vfill
      \vspace{1em}
      \url{\repo}
      \vfill
      \vspace{1em}
      Para rodar os testes dos slides:
      \mintinline{shell}{python3.8 -m doctest slides.tex}
      \vfill
      \vspace{1em}
    \end{center}
    \vspace{-1em}
    Referências:
    \column{.5\textwidth}
    \begin{center}
      \qrcode[height=5cm]{\repo}
    \end{center}
  \end{columns}
  \vfill
  \fontsize{8pt}{7pt}\selectfont
  \begin{itemize}[label=--, leftmargin=0mm]
    \item \url{https://bugs.python.org/issue1714448}
    \item \url{https://www.python.org/dev/peps/pep-0379/}
    \item \url{https://www.python.org/dev/peps/pep-0572/}
    \item \url{https://github.com/python/peps/commits%
                      /master?path[]=pep-0572.rst}
    \item \url{https://mail.python.org/pipermail/python-ideas%
                      /2018-February/049041.html}
    \item \url{https://mail.python.org/pipermail/python-ideas%
                      /2009-March/003423.html}
    \item \url{https://mail.python.org/pipermail/python-ideas%
                      /2009-March/003439.html}
    \item \url{https://mail.python.org/pipermail/python-ideas%
                      /2018-March/049408.html}
    \item \url{https://mail.python.org/pipermail/python-ideas%
                      /2018-March/049409.html}
    \item \url{https://mail.python.org/pipermail/python-ideas%
                      /2018-April/049758.html}
    \item \url{https://mail.python.org/pipermail/python-ideas%
                      /2018-April/049957.html}
    \item \url{https://mail.python.org/pipermail/python-ideas%
                      /2018-May/050456.html}
    \item \url{https://mail.python.org/pipermail/python-dev%
                      /2018-July/154601.html}
    \item \url{https://mail.python.org/pipermail/python-committers%
                      /2018-July/005664.html}
  \end{itemize}
\end{frame}


\end{document}
